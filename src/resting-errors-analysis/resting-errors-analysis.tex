\documentclass{article}
\usepackage{amsmath}
\usepackage{mathtools}
\usepackage{tikz}
\usepackage{bm}
\usepackage[colorlinks,linkcolor=blue,citecolor=blue,urlcolor=blue]{hyperref}
\usepackage{siunitx}
\usepackage{eufrak}
\usepackage{filecontents}
\usepackage{natbib}
\usepackage{doi}

\begin{filecontents}{resting-errors-analysis.bib}
@article{shaw-weller2016,
	title={Comparison of Terrain Following and Cut Cell Grids using a Non-Hydrostatic Model},
	author={Shaw, James and Weller, Hilary},
	journal={Mon.\ Wea.\ Rev.},
	year={2016},
	volume={144},
	number={6},
	pages={2085--2099},
	doi={10.1175/MWR-D-15-0226.1}
}

@article{weller-shahrokhi2014,
	title={Curl free pressure gradients over orography in a solution of the fully compressible {Euler} equations with implicit treatment of acoustic and gravity waves},
	author={Hilary Weller and Ava Shahrokhi},
	journal={Mon.\ Wea.\ Rev.},
	year={2014},
	volume={142},
	number={12},
	pages={4439--4457},
	doi={10.1175/MWR-D-14-00054.1}
}
\end{filecontents}

\newcommand{\Mag}[1]{\lvert #1 \rvert}
\newcommand{\vect}{\mathbf}
\newcommand{\Exner}{\Pi}
\newcommand{\Sf}{\vect{S}_f}
\newcommand{\moment}{\mathfrak{m}}
\newcommand{\iu}{{i\mkern1mu}}

\title{Numeric errors associated with the balance between pressure gradient and gravity terms}
\author{James Shaw}

\begin{document}
\maketitle
In the resting atmosphere tests in section 3.4 of \href{http://datumedge.co.uk/publications/phd-thesis.pdf}{the thesis}, unstable results are obtained on terrain-following meshes with very steep slopes.  Hilary noted in \href{https://github.com/hertzsprung/thesis/blob/master/src/thesis-reviews/weller-2017-10-26.ps.gz}{her 2017-10-26 review} that errors are a result of a numerical imbalance between the pressure gradient and gravity.  This document discusses some thoughts on the nature of these errors.

The discretisation of the fully compressible Euler equations by \citet{weller-shahrokhi2014} prognoses the Exner function of pressure, $\Exner$, stored at cell centres.
The pressure gradient term is discretised by calculating Exner gradients in the direction of adjacent cell centres.  As such, gradients are entirely horizontal or entirely vertical only when the mesh is orthogonal.
On orthogonal meshes, the pressure gradient at vertical faces is exactly zero assuming horizontally uniform, stably-stratified initial conditions.
In the resting atmosphere test, the initial $\Exner$ field is calculated to obtain discrete hydrostatic balance with the initial potential temperature field, so pressure gradients are zero on all faces.
Terrain-following meshes are non-orthogonal and the direction of adjacent cell centres is misaligned with surfaces of constant gravitational potential.
Hence, on terrain-following meshes, the pressure gradient on vertical faces is non-zero.
The gravity term, which is also calculated in the direction between cell centres, should be cancel the pressure gradient.  Otherwise, cancellation errors may arise between the pressure gradient and gravity terms (see equation (14) in \citet{weller-shahrokhi2014}).  Might such errors become larger over steeper terrain? Why?

A cancellation error between the pressure gradient and gravity terms is one possible cause for the unstable results obtained on steep terrain-following meshes.  There are two other possible causes.  First, while $\nabla \Exner$ is formulated to be curl-free, it is not necessarily the case that the pressure gradient $\nabla p$ is curl-free \citep{weller-shahrokhi2014}.
Second, the orthogonal part of the pressure gradient is calculated implicitly, the non-orthogonal part is a deferred correction (see the appendix to \citet{shaw-weller2016}.
The resting atmosphere test became unstable around $t = \SI{50}{\second}$ on a basic terrain-following mesh and a peak mountain height $h_0 = \SI{6}{\kilo\meter}$.
So perhaps the system of equations is not converging for highly non-orthogonal meshes?  Reducing the timestep to resolve acoustic waves should ensure convergence because the implicit treatment becomes redundant.  However, reducing the timestep from $\Delta t = \SI{25}{\second}$ to $\Delta t = \SI{0.01}{\second}$ did not help, so I'm not convinced that a lack of convergence is the cause of the instability.

\bibliographystyle{ametsoc2014}
\bibliography{resting-errors-analysis}

\end{document}
