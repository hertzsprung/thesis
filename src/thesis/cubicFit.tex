\chapter{Numerically stable transport over steep slopes}
\label{ch:cubicFit}

\begin{highlights}
{\Large Highlights}
\begin{itemize}
	\item The new cubicFit transport scheme is second-order convergent regardless of mesh distortions or the choice of velocity field
	\item Sub-grid reconstructions are mostly precomputed depending on the mesh geometry alone
	\item Misalignment of the velocity field with mesh layers is the primary source of numerical error, not simply mesh distortions
\end{itemize}
\end{highlights}

\TODO{
Motivation
\begin{itemize}
	\item Conservation of tracer, therefore Eulerian
	\item Numerically stable and accurate on arbitrary meshes (because no consensus on "best" mesh: hex or cubed-sphere? TF or cut cells?)
	\item Avoid all splitting errors (because they get worse with steeper slopes represented by TF meshes)
	\item Method-of-lines approach permits low computational cost
\end{itemize}
}

%Numerical transport schemes have a number of desirable properties: 
% - numerical
%   - conservation of tracer
%   - numerical stability
%   - low numerical diffusion and dispersion
%   - better than first-order convergence (staniforth-thuburn2012)
%   - lack of grid imprinting (staniforth-thuburn2012)
%   - insensitive to mesh distortions or jumps in mesh spacing (e.g. AMR)
% - computational
%   - time-to-solution

\section{Transport schemes for arbitrary meshes}
\label{sec:cubicFit:transport}

The transport of a tracer density $\phi$ in a prescribed, non-divergent velocity field $\vect{u}$ is given by the flux-form equation \citep{nair-lauritzen2010}
\begin{align}		
	\frac{\partial \phi}{\partial t} + \nabla \cdot \left( \vect{u} \phi \right) = 0 \text{ .} \label{eqn:advection}		
\end{align}
The time derivative is discretised using an explicit, two-stage, second-order Heun scheme,
\begin{subequations}
\begin{align}
	\phi^\star &= \phi^{(n)} + \Delta t \: g(\phi^{(n)}) \\
	\phi^{(n+1)} &= \phi^{(n)} + \frac{\Delta t}{2} \left[ g(\phi^{(n)}) + g(\phi^{\star}) \right]
\end{align} \label{eqn:heun}%
\end{subequations}
\unskip where \(g(\phi^{(n)}) = - \nabla \cdot (\vect{u} \phi^{(n)})\) at time level \(n\).
This two-stage second-order time-stepping scheme is similar to the three-stage second-order time-stepping scheme used later in a model of the fully compressible Euler equations (section~\ref{sec:slanted:exnerFoamH}), which needs an additional time-stage to converge upon the semi-implicit solution.
The two-stage second-order time-stepping scheme is used for both the cubicFit scheme and the multidimensional linear upwind scheme.
Although the Heun scheme is unstable for a linear oscillator \citep{durran2013} and for solving the transport equation using centred, linear differencing, it is stable when it is used for transport schemes with sufficient upwinding \citep[p. 149]{hundsdorfer-verwer2013}.

Using the finite volume method, the velocity field is prescribed at face centroids and the dependent variable is stored at cell centroids.  The divergence term in equation~\eqref{eqn:advection} is discretised using Gauss's theorem,
\begin{align}
	\nabla \cdot \left( \vect{u} \phi \right) \approx \frac{1}{\vol_c} \sum_{f \in\:c} \vect{u}_f \cdot \Sf \phi_F \label{eqn:gauss-div}
\end{align}
where subscript $f$ denotes a value stored at a face and subscript $F$ denotes a value approximated at a face from surrounding values.  $\vol_c$ is the cell volume, $\vect{u}_f$ is a velocity vector prescribed at a face, $\Sf$ is the surface area vector with a direction outward normal to the face and a magnitude equal to the face area, $\phi_F$ is an approximation of the dependent variable at the face, and $\sum_{f \in\:c}$ denotes a summation over all faces $f$ bordering cell $c$.

This discretisation is applicable to arbitrary meshes.  A necessary condition for stability is given by the multidimensional Courant number \citep{weller-shahrokhi2014},
\begin{align}
	\mathrm{Co}_c = \frac{\Delta t}{2 \mathcal{V}_c} \sum_{f \in\: c} \Mag{\vect{u} \cdot \Sf} \label{eqn:co}
\end{align}
such that, for all cells $c$ in the domain, $\mathrm{Co}_c$ is less than or equal to some constant that depends upon the spatial and temporal discretisation.  Hence, stability is constrained by the maximum Courant number of any cell in the domain.

The accurate approximation of the dependent variable at the face, $\phi_F$, is key to the overall accuracy of the transport scheme. The cubicFit scheme and the multidimensional linear upwind scheme differ in their approximations, and these approximation methods are described next.


\section{High-order finite volume formulation}
\label{sec:highOrderFit:scheme}

Integrating the flux-form transport equation \eqref{eqn:advection} over a volume $\vol$ and using Gauss's divergence theorem,
\begin{align}
	\int_\vol \frac{\partial \phi}{\partial t} d\vol = - \int_{\partial \vol} \phi \vect{u} \cdot \unitn dA \label{eqn:highOrder:integrated-advection}
\end{align}
where $\unitn$ is the outward unit normal vector.
Using the method of lines, the time derivative is discretised using the classical fourth-order Runge–Kutta time-stepping scheme \citep[p. 53]{durran2013}, and the spatial discretisation is described next.
For a polygonal cell with faces $f$ equation~\eqref{eqn:highOrder:integrated-advection} becomes
\begin{align}
	\int_\vol \frac{\partial \phi}{\partial t} d\vol = - \sum_f \int_{\area_f} \phi \vect{u} \cdot \unitn d\area_f \label{eqn:highOrder:advection}
\end{align}
where $\area_f$ is the area of face $f$.
If $\phi$ is a sufficiently smooth field then it can be approximated to $P$-order accuracy by replacing $\phi$ with a polynomial interpolant $\psi$,
\begin{align}
	\psi = \sum_{\Mag{\vect{p}} \leq P} a_{\vect{p}} \left(\vect{x} - \vect{x}_0 \right)^\vect{p} \label{eqn:highOrder:interpolant}
\end{align}
where $a_\vect{p}$ are unknown polynomial coefficients, $\vect{x}_0$ is a fixed position, and $P$ is the total polynomial order.
Note that we use the multi-index notation such that $\Mag{\vect{p}} = p_1 + \ldots + p_n$ and
\begin{align}
	a_{\vect{p}} \left( \vect{x} - \vect{x}_0 \right)^\vect{p} = a_\vect{p} \prod_{d=1}^D \left( x_d - x_{0_d} \right)^{p_d} \text{ .}
\end{align}
where $D$ is the number of physical dimensions.
As an example, the exponents $\vect{p}$ in two dimensions $(x, y)$ with $\Mag{\vect{p}} \leq 1$ are $(0, 0)$, $(1, 0)$ and $(0, 1)$, hence the two-dimensional polynomial interpolant for a total polynomial order $P = 1$ is
\begin{align}
	\psi = a_{0,0} + a_{1,0} \left( x - x_0 \right) + a_{0,1} \left( y - y_0 \right) \text{ .}
\end{align}
Replacing $\phi$ in \eqref{eqn:highOrder:advection} with $\psi$ in \eqref{eqn:highOrder:interpolant} we obtain an expression for the face flux,
\begin{align}
	\int_\area \phi \vect{u} \cdot \unitn d\area = \uf \cdot \unitn \sum_{\Mag{\vect{p}} \leq P} a_{\vect{p}} \moment_\area^\vect{p} \label{eqn:highOrder:face-flux}
\end{align}
where $\moment_\area^\vect{p} = \int_\area \left( \vect{x} - \vect{x}_0 \right)^\vect{p} d \area$ is the $\vect{p}$-th moment of area $\area$.
\TODO{here we're assuming that $\vect{u}$ is smoother than $\phi$, see \citet{methven-hoskins1999} for a justification?}
Therefore, the face flux can be calculated by finding the the polynomial coefficients $a_\vect{p}$.

Following the same approach as the cubicFit transport scheme, taking a total polynomial order $P = 3$ gives 9 polynomial terms with polynomial coefficients calculated using the same upwind-biased stencil.
For every cell in the stencil we require that the average of the polynomial integrated over a cell volume equals the cell average value,
\begin{align}
	\volave{\psi} = \volave{\phi} \label{eqn:highOrder:equal-volumes}
%
\intertext{where the average over volume $\vol$ is}
%
	\volave{\psi} = \frac{1}{\vol} \int_\vol \psi d\vol \text{ .} \label{eqn:highOrder:volume-average}
\end{align}
Using equations~\eqref{eqn:highOrder:interpolant} and \eqref{eqn:highOrder:volume-average} we can rewrite equation~\eqref{eqn:highOrder:equal-volumes} as
\begin{align}
	\frac{1}{\moment_\vol^\vect{0}} \sum_{\Mag{\vect{p}} \leq P} a_\vect{p} \moment_\vol^\vect{p} = \volave{\phi}
\end{align}
where $\moment_\vol^\vect{p} = \int_\vol \left( \vect{x} - \vect{x}_0 \right)^\vect{p} d\vol$ is the $\vect{p}$-th moment of volume $\vol$, and the zeroth moment $\moment_\vol^\vect{0}$ is the volume.
For $m$ polynomial terms and a stencil with $n$ cells, we calculate a face flux by choosing $\vect{x}_0$ to be the position of the face centre, then we write the linear system
\begin{align}
	\begin{bmatrix}
		\moment_{\vol_1}^{\vect{p}_1}/\moment_{\vol_1}^\vect{0} & \cdots & \moment_{\vol_1}^{\vect{p}_m}/\moment_{\vol_1}^\vect{0} \\
		\vdots & & \vdots \\
		\moment_{\vol_n}^{\vect{p}_1}/\moment_{\vol_n}^\vect{0} & \cdots & \moment_{\vol_n}^{\vect{p}_m}/\moment_{\vol_n}^\vect{0}
	\end{bmatrix}
	\begin{bmatrix}
		a_{\vect{p}_1} \\
		\vdots \\
		a_{\vect{p}_m}
	\end{bmatrix}
	=
	\begin{bmatrix}
		\langle \phi \rangle_{\vol_1} \\
		\vdots \\
		\langle \phi \rangle_{\vol_n}
	\end{bmatrix} \label{eqn:highOrder:moment-matrix}
\end{align}
which can be written as
\begin{align}
	\vect{B} \vect{a} = \bm{\phi} \text{ .}
\end{align}
where $\vect{B}$ is the stencil matrix, which is constructed using only the mesh geometry.
The highOrderFit scheme generates stencils using the same procedure as the cubicFit scheme.
Assuming a stencil comprises at least as many cells as there are polynomial coefficients then $n \geq m$ and the matrix equation can be solved using a least-squares approach to find the unknown coefficients $\vect{a}$.

To obtain a stable transport scheme, we follow the approach of the cubicFit scheme by introducing multipliers $\vect{m}$ to obtain
\begin{align}
	\vect{\tilde{B}} \vect{a} = \vect{m} \cdot \bm{\phi}
\end{align}
where $\vect{\tilde{B}} = \vect{M} \vect{B}$ and $\vect{M} = \mathrm{diag}(\vect{m})$.
The upwind cell and downwind cell have multipliers $m_u = 2^{10}$ and $m_d = 2^{10}$ respectively, and all peripheral points have multipliers $m_p = 1$.

The calculation of high-order cell volume moments and surface moments are required by equations~\eqref{eqn:highOrder:moment-matrix} and~\eqref{eqn:highOrder:face-flux} respectively.  These volume and surface moments can be calculated exactly using the method of \citet{tuzikov2003}.
We follow their method but, in order to avoid any degenerate triangles, we introduce a centre point shared by all triangles instead of triangulating polygons with only existing vertices.

While the highOrderFit transport scheme uses a total polynomial order $P = 3$ for stencils in the domain interior, a total polynomial order $P = 1$ is used for stencils near the boundary having fewer than 12 cells.
This reduction in total polynomial order ensures that matrix equations are never underconstrained.
This thesis does not assess the accuracy of the highOrderFit scheme near boundaries, and so the more sophisticated boundary treatment implemented in the cubicFit scheme has not been implemented in the highOrderFit scheme.


\subsection{Multidimensional linear upwind transport scheme}

The multidimensional linear upwind scheme, called ``linearUpwind'' hereafter, is documented here since it provides a baseline accuracy for the experiments that follow.  The approximation of $\phi_F$ is calculated using a gradient reconstruction,
\begin{align}
	\phi_F &= \phi_u + \nabla_c\: \phi \cdot \left(\vect{x}_f - \vect{x}_c \right)
\end{align} 
where $\phi_u$ is the upwind value of $\phi$, and $\vect{x}_f$ and $\vect{x}_c$ are the position vectors of the face centroid and cell centroid respectively.
The gradient $\nabla_c \:\phi$ is calculated using Gauss's theorem:
\begin{align}
	\nabla_c\: \phi = \frac{1}{\vol_c} \sum_{f\in\:c} \widetilde{\phi}_F \Sf \label{eqn:linearUpwind-grad}
\end{align}
where $\widetilde{\phi}_F$ is linearly interpolated from the two neighbouring cells of face $f$.
The resulting stencil comprises all cells sharing a face with the upwind cell, including the upwind cell itself.  For a face in the interior of a two-dimensional rectangular mesh, the stencil for the linearUpwind scheme is a `$+$' shape with 5 cells.  On the same mesh, the stencil for the cubicFit scheme is more than twice the size with 12 cells.
For cells adjacent to boundaries having zero gradient boundary conditions, the boundary value is set to be equal to the cell centre value before equation~\eqref{eqn:linearUpwind-grad} is evaluated.
This implementation of the multidimensional linear upwind scheme is included with OpenFOAM \citep{openfoam-numerics}.

\section{Horizontal transport over mountains}
\label{sec:cubicFit:schaerAdvect}

A two-dimensional transport test was developed by \citet{schaer2002} to study the effect of terrain-following coordinate transformations on numerical accuracy.
In this standard test, a tracer is positioned aloft and transported horizontally over wave-shaped mountains.
When terrain-following meshes are used, this test challenges transport schemes because the tracer must cross mesh layers, which acts to reduce numerical accuracy \citep{schaer2002}.
Here we use a more challenging variant of the test that has steeper mountains and highly-distorted terrain-following meshes.
Numerical convergence and numerical error structures are compared using the linearUpwind and cubicFit transport schemes on terrain-following meshes and cut cell meshes.

The domain is defined on a rectangular $x$--$z$ plane that is \SI{300}{\kilo\meter} wide as measured between the outermost cell centres, and \SI{25}{\kilo\meter} high as measured between upper and lower boundary edges.
Boundary conditions are imposed on the tracer density $\phi$ such that $\phi = \SI{0}{\kilo\gram\per\meter\cubed}$ at the inlet boundary, and a zero normal gradient
$\partial \phi / \partial n = \SI{0}{\kilo\gram\per\meter\tothe{4}}$ is imposed at the outlet boundary.  There is no normal flow at the lower and upper boundaries.

The terrain is wave-shaped, specified by the surface elevation $h$ such that
\begin{subequations}
\begin{align}
   h(x) &= h^\star \cos^2 ( \alpha x )
%
\shortintertext{where}
%
   h^\star(x) &= \left\{ \begin{array}{l l}
       h_0 \cos^2 ( \beta x ) & \enskip \text{if $| x | < a$} \\
	0 & \enskip \text{otherwise}
    \end{array} \right.
\end{align}
\end{subequations}
where $a = \SI{25}{\kilo\meter}$ is the mountain envelope half-width, $h_0 = \SI{6}{\kilo\meter}$ is the maximum mountain height, $\lambda = \SI{8}{\kilo\meter}$ is the wavelength, \(\alpha = \pi / \lambda\) and \(\beta = \pi / (2a)\).  Note that, in order to make this test more challenging, the mountain height $h_0$ is double the mountain height used by \citet{schaer2002}.

A basic terrain-following (BTF) mesh is constructed by using the terrain profile to modify the uniform rectangular mesh.
The BTF method uses a linear decay function so that mesh layers become horizontal at the top of the model domain \citep{galchen-somerville1975a},
\begin{equation}
	z(x) = \left( H - h(x) \right) \left( z^\star / H \right) + h(x) \label{eqn:btf}
\end{equation}
where $z$ is the geometric height, $H$ is the height of the domain, $h(x)$ is the surface elevation and $z^\star$ is the computational height of a mesh layer.  If there were no terrain then $h = 0$ and $z = z^\star$.

A velocity field is prescribed with uniform horizontal flow aloft and zero flow near the ground,
\begin{align}
	u(z) = u_0 \left\{ \begin{array}{l l}
		1 & \enskip \text{if $z \geq z_2$} \\
		\sin^2 \left( \frac{\pi}{2} \frac{z - z_1}{z_2 - z_1} \right) & \enskip \text{if $z_1 < z < z_2$} \\
		0 & \enskip \text{otherwise}
	\end{array} \right.	
\end{align}
where $u_0 = \SI{10}{\meter\per\second}$, $z_1 = \SI{7}{\kilo\meter}$ and $z_2 = \SI{8}{\kilo\meter}$.
This results in a constant wind above $z_2$, and zero flow at \SI{7}{\kilo\meter} and below.

The discrete velocity field is defined using a streamfunction, \(\Psi\).  Given that \(u = -\partial \Psi / \partial z\), the streamfunction is found by vertical integration of the velocity profile:
\begin{align}
	\Psi(z) &= -\frac{u_0}{2} \left\{ \begin{array}{l l}
		\left( 2z - z_1 - z_2 \right) & \enskip \text{if $z > z_2$} \\
		z - z_1 - \frac{z_2 - z_1}{\pi} \sin \left(\pi \frac{z - z_1}{z_2-z_1}\right) & \enskip \text{if $z_1 < z \leq z_2$} \\
		0 & \enskip \text{if $z \leq z_1$}
	\end{array} \right.
\end{align}

A tracer with density $\phi$ is positioned upstream above the height of the terrain.  It has the shape
\begin{subequations}
\begin{align}
	\phi(x, z) &= \phi_0 \left\{ \begin{array}{l l}
		\cos^2 \left( \frac{\pi r}{2} \right) & \enskip \text{if $r \leq 1$} \\
		0 & \enskip \text{otherwise}
	\end{array} \right.
%
\intertext{with radius $r$ given by}
%
	r &= \sqrt{
		\left( \frac{x - x_0}{A_x} \right)^2 + 
		\left( \frac{z - z_0}{A_z} \right)^2
	}
\end{align}
%
\label{eqn:cubicFit:schaerAdvect:tracer}
\end{subequations}
where $A_x = \SI{25}{\kilo\meter}$, $A_z = \SI{3}{\kilo\meter}$ are the horizontal and vertical half-widths respectively, and $\phi_0 = \SI{1}{\kilogram\per\meter\cubed}$ is the maximum density of the tracer.  At $t = \SI{0}{\second}$, the tracer is centred at $(x_0, z_0) = (\SI{-50}{\kilo\meter}, \SI{12}{\kilo\meter})$ so that the tracer is upwind of the mountain, in the region of uniform flow above $z_2$.

Tests are integrated for \SI{10000}{\second} using s chosen for each mesh so that the maximum Courant number is about \num{0.4}.  This choice yields a time-step that is well below any stability limit, as recommended by \citet{lauritzen2012}.  By the end of integration the tracer is positioned downwind of the mountain.
The analytic solution at $t = \SI{10000}{\second}$ is centred at $(x_0, z_0) = (\SI{50}{\kilo\meter}, \SI{12}{\kilo\meter})$ with its shape unchanged from the initial condition.

\begin{figure}
	\centering
	\begin{subfigure}{\textwidth}
		\input{cubicFit/schaerAdvect/convergence}
		\phantomsubcaption\label{fig:cubicFit:schaerAdvect:convergence:l2}
		\phantomsubcaption\label{fig:cubicFit:schaerAdvect:convergence:linf}
		\phantomsubcaption\label{fig:cubicFit:tfAdvect:convergence:l2}
		\phantomsubcaption\label{fig:cubicFit:tfAdvect:convergence:linf}
	\end{subfigure}
%
	\caption{Numerical convergence of the two-dimensional tracer transport tests over mountains using
	(\subcaptionref{fig:cubicFit:schaerAdvect:convergence:l2}, \subcaptionref{fig:cubicFit:schaerAdvect:convergence:linf}) horizontal and
	(\subcaptionref{fig:cubicFit:tfAdvect:convergence:l2}, \subcaptionref{fig:cubicFit:tfAdvect:convergence:linf}) terrain-following velocity fields.
	$\ell_2$ errors (equation~\ref{eqn:l2-error}) and $\ell_\infty$ errors (equation~\ref{eqn:linf-error}) are marked at mesh spacings between \SI{5000}{\meter} and \SI{250}{\meter} using linearUpwind and cubicFit transport schemes on basic terrain-following and cut cell meshes.}
	\label{fig:cubicFit:schaerAdvect:convergence}
\end{figure}

To assess numerical convergence, a range of mesh spacings are chosen so that $\Delta x \mathbin{:} \Delta z = 2\mathbin{:}1$ to match the original test specification from \citet{schaer2002}.
Tests were performed using the linearUpwind and cubicFit schemes using BTF meshes and cut cell meshes with mesh spacings between $\Delta x = \SI{250}{\meter}$ and $\Delta x = \SI{5000}{\meter}$.
Error norms are calculated by subtracting the analytic solution from the numerical solution,
\begin{align}
	\ell_2 &= \sqrt{\frac{\sum_c \left(\phi - \phi_T \right)^2 \vol_c}{\sum_c \left(\phi_T^2 \vol_c \right)}} \label{eqn:l2-error} \\
	\ell_\infty &= \frac{\max_c \Mag{\phi - \phi_T}}{\max_c \Mag{\phi_T}} \label{eqn:linf-error}
\end{align}
where $\phi$ is the numerical value, $\phi_T$ is the analytic value, $\sum_c$ denotes a summation over all cells $c$ in the domain, and $\max_c$ denotes a maximum value of any cell.
The linearUpwind and cubicFit schemes are second-order convergent in the $\ell_2$ norm (figure~\ref{fig:cubicFit:schaerAdvect:convergence:l2}) and $\ell_\infty$ norm (figure~\ref{fig:cubicFit:schaerAdvect:convergence}) at all but the coarsest mesh spacings where errors are saturated for both schemes.

The cubicFit scheme achieves a given $\ell_2$ error using a mesh spacing that is almost twice as coarse as that needed by the linearUpwind scheme.  Doubling the mesh spacing results in a coarser mesh with four times fewer cells because the $\Delta x \mathbin{:} \Delta z$ aspect ratio is fixed.
Recall that the stencil for the cubicFit scheme has about twice as many cells as the stencil for the linearUpwind scheme.
Hence, for a given $\ell_2$ error, the computational cost during integration of the cubicFit scheme is about half the computational cost of the linearUpwind scheme.

\begin{figure}
	\centering
	\begin{subfigure}{\textwidth}
		\centering
		\includegraphics{thesis/cubicFit/schaerAdvect/fig-error.pdf}
		\phantomsubcaption\label{fig:cubicFit:schaerAdvect:error:btf:linearUpwind}
		\phantomsubcaption\label{fig:cubicFit:schaerAdvect:error:cutCell:linearUpwind}
		\phantomsubcaption\label{fig:cubicFit:schaerAdvect:error:btf:cubicFit}
		\phantomsubcaption\label{fig:cubicFit:schaerAdvect:error:cutCell:cubicFit}
		\phantomsubcaption\label{fig:cubicFit:tfAdvect:error:btf:linearUpwind}
		\phantomsubcaption\label{fig:cubicFit:tfAdvect:error:cutCell:linearUpwind}
		\phantomsubcaption\label{fig:cubicFit:tfAdvect:error:btf:cubicFit}
		\phantomsubcaption\label{fig:cubicFit:tfAdvect:error:cutCell:cubicFit}
	\end{subfigure}
	\caption{Tracer contours at the end of integration for the two-dimensional tracer transport tests over mountains using
	(\subcaptionref{fig:cubicFit:schaerAdvect:error:btf:linearUpwind},
	\subcaptionref{fig:cubicFit:schaerAdvect:error:cutCell:linearUpwind},
	\subcaptionref{fig:cubicFit:schaerAdvect:error:btf:cubicFit},
	\subcaptionref{fig:cubicFit:schaerAdvect:error:cutCell:cubicFit}) horizontal and 
	(\subcaptionref{fig:cubicFit:tfAdvect:error:btf:linearUpwind},
	\subcaptionref{fig:cubicFit:tfAdvect:error:cutCell:linearUpwind},
	\subcaptionref{fig:cubicFit:tfAdvect:error:btf:cubicFit},
	\subcaptionref{fig:cubicFit:tfAdvect:error:cutCell:cubicFit}) terrain-following velocity fields.  The numerical solution, marked with solid lines, and the analytic solution, marked with dashed lines, are plotted every \num{0.1}.  Tracer contours overlay a color error field, calculated by subtracting the analytic solution from the numerical solution.  Only the lowest \SI{20}{\kilo\meter} in the lee of the mountain is plotted.  The entire domain is \SI{300}{\kilo\meter} wide and \SI{25}{\kilo\meter} high.
	}
	
	\label{fig:cubicFit:schaerAdvect:error}
\end{figure}

Next, we examine the structure of numerical errors with test results using the linearUpwind and cubicFit transport schemes on BTF and cut cell meshes with $\Delta x = \SI{1000}{\meter}$ and $\Delta z = \SI{500}{\meter}$.
To obtain a maximum Courant number of about \num{0.4}, we choose $\Delta t = \inputval{schaerAdvect-cutCell-1000-linearUpwind/dt}$ on the cut cell mesh where the flow is aligned with mesh layers and there are no fluxes through upper and lower cell faces.
Since there is no flow below $z = \SI{7}{\kilo\meter}$, the time-step is not constrained by small, cut cells next to the lower boundary.
On the BTF mesh, $\Delta t$ is only \inputval{schaerAdvect-btf-1000-linearUpwind/dt} because the flow is misaligned with mesh layers, with fluxes through all four faces of cells above sloping terrain.

The highly-distorted BTF mesh presents a particular challenge to the linearUpwind scheme with the final numerical solution, marked by solid lines, losing its correct shape and maximum intensity compared to the analytic solution marked by dashed lines (figure~\ref{fig:cubicFit:schaerAdvect:error:btf:linearUpwind}).
The linearUpwind scheme produces a much better solution on the cut cell mesh, with only small phase errors apparent in figure~\ref{fig:cubicFit:schaerAdvect:error:cutCell:linearUpwind}.
Accuracy is much improved using the cubicFit scheme: on the BTF mesh, shape and maximum intensity are similar to the analytic solution (figure~\ref{fig:cubicFit:schaerAdvect:error:btf:cubicFit}) and, on the cut cell mesh, numerical errors are so small they are not visible (figure~\ref{fig:cubicFit:schaerAdvect:error:cutCell:cubicFit}).
The numerical and analytic contours overlay a color error field that reveals horizontal streaks of error on the BTF mesh (figure~\ref{fig:cubicFit:schaerAdvect:error:btf:linearUpwind},~\ref{fig:cubicFit:schaerAdvect:error:btf:cubicFit}) that were generated above the steepest mountain peaks before becoming trapped in the region of zero flow below $z = \SI{7}{\kilo\meter}$.

The horizontal transport test demonstrates that the cubicFit scheme is second-order convergent in the domain interior irrespective of mesh distortions.  Numerical errors are largest on terrain-following meshes, due either to misalignment of the flow with mesh layers, or to mesh distortions.
In the next section, we propose a new test in order to identify the primary cause of these numerical errors.

\section{Transport in a terrain-following velocity field}

\TODO{
\begin{itemize}
	\item Conclusion: Misalignment of the velocity field with mesh layers is the primary source of numerical error
	\item Conclusion: 2nd-order convergence again
\end{itemize}
}

In the horizontal transport test, results were least accurate on the BTF mesh where the mesh was most distorted and flow was misaligned with the mesh layers.
Here, we formulate a new tracer transport test in which the velocity field is everywhere tangential to the basic terrain-following mesh layers.
The flow is then aligned with the BTF mesh layers, but the points in the linearUpwind and cubicFit stencils are not uniformly distributed because the BTF mesh is distorted.
Conversely, the flow is misaligned with the cut cell mesh layers but, except in cut cells next to the ground, the cut cell mesh is undistorted.
This test determines whether the primary source of numerical error is due to mesh distortions or misalignment of the flow with mesh layers.

The spatial domain, mountain profile, initial tracer profile and discretisation are the same as those in the horizontal tracer advection test, \TODO{except for the choice of timesteps... what are the timesteps?}
The discrete velocity field is calculated using a streamfunction in the same way as the horizontal transport test.
Here, we define a streamfunction $\Psi$ is that provides a velocity field that follows the BTF mesh layers given by equation~\eqref{eqn:btf} such that
\begin{equation}
	\Psi(x,z) = -u_0 H_1 \frac{z - h}{H_1 - h} \label{eqn:streamfunc-btf}
\end{equation}
where $u_0 = \SI{10}{\meter\per\second}$, which is the horizontal velocity where $h(x) = 0$.
The velocity field follows the lower boundary and becomes entirely horizontal at $H_1 = H = \SI{25}{\kilo\meter}$, hence, there is no normal flow at the lower and upper boundaries.
In the domain interior, the flow is predominantly horizontal, with non-zero vertical velocities only above sloping terrain.

The horizontal and vertical components of velocity, $u$ and $w$, are given by
\begin{align}
	u &= -\frac{\partial \Psi}{\partial z} = u_0 \frac{H_1}{H_1 - h}, \quad w = \frac{\partial \Psi}{\partial x} = u_0 H_1 \frac{\mathrm{d} h}{\mathrm{d} x} \frac{H_1 - z}{\left( H_1 - h \right)^2} \label{eqn:slanted:uw-btf} \text{ ,}\\
	\frac{\mathrm{d} h}{\mathrm{d} x} &= - h_0 \left[ 
		\beta \cos^2 \left( \alpha x \right) \sin \left( 2 \beta x \right) +
		\alpha \cos^2 \left( \beta x \right) \sin \left( 2 \alpha x \right)
	\right] \text{ .}
\end{align}
Unlike the horizontal transport test, the velocity field presented here extends from the top of the domain all the way to the ground.

An analytic solution at \SI{10000}{\second} is obtained by calculating the new horizontal position of the tracer.  Integrating along the trajectory yields $t$, the time taken to move from the left side of the mountain at $-a$, to the right side of the mountain at $a$,
\begin{align}
	\mathrm{d}t &= \mathrm{d}x / u(x) \\
	t &= \int_{-a}^a \frac{H_1 - h(x)}{u_0 H_1}\:\mathrm{d}x \\
	t &= \left. \frac{2a}{u_0} - \frac{h_0}{16 u_0 H_1} \left[ 4x + \frac{\sin 2 (\alpha + \beta) x}{\alpha + \beta} +
\frac{\sin 2(\alpha - \beta) x}{\alpha - \beta} + 2 \left( \frac{\sin 2\alpha x}{\alpha} + \frac{\sin 2\beta x}{\beta} \right) \right] \right\rvert_{-a}^a
\end{align}
Because the velocity field is non-divergent, the flow accelerates over mountain ridges and the tracer travels \TODO{\SI{123}{\meter}} further compared to the tracer in a purely horizontal velocity field.  Tracer height is unchanged downwind of the mountains because flow is parallel to the mesh layers.

\begin{figure}
	\centering
	\includegraphics{thesis/cubicFit/tfAdvect/fig-tracer.pdf}
	\caption{\TODO{tracer evolution with BTF cubicFit}}
	\label{fig:cubicFit:tfAdvect:tracer}
\end{figure}

\section{Deformational flow on a sphere}

\subsection{Transport under divergent flow conditions using cosine bells}
\label{sec:deformationSphere:divergent}

\TODO{Velocity field specified by \citet{nair-lauritzen2010}}
\begin{align}
	u(\lon, \lat, t) &= -5 \frac{\Rearth}{T} \sin^2 \left( \frac{\lon'}{2} \right) \sin \left(2 \lat \right) \cos^2 \left( \lat \right) \cos \left( \frac{\pi t}{T} \right) + \frac{2 \pi \Rearth}{T} \cos \left( \lat \right) \text{,} \\
	v(\lon, \lat, t) &= \frac{5}{2} \frac{\Rearth}{T} \sin \left( \lon' \right) \cos^3 \left( \lat \right) \cos \left( \frac{\pi t}{T} \right)
\end{align}



