\section{Horizontal transport over mountains}
\label{sec:cubicFit:schaerAdvect}

A two-dimensional transport test was developed by \citet{schaer2002} to study the effect of terrain-following coordinate transformations on numerical accuracy.
In this standard test, a tracer is positioned aloft and transported horizontally over wave-shaped mountains.
When terrain-following meshes are used, this test challenges transport schemes because the tracer must cross mesh layers, which acts to reduce numerical accuracy \citep{schaer2002}.
Here we use a more challenging variant of the test that has steeper mountains and highly-distorted terrain-following meshes.
Numerical convergence and numerical error structures are compared using the linearUpwind and cubicFit transport schemes on terrain-following meshes and cut cell meshes.

The domain is defined on a rectangular $x$--$z$ plane that is \SI{300}{\kilo\meter} wide as measured between the outermost cell centres, and \SI{25}{\kilo\meter} high as measured between upper and lower boundary edges.
Boundary conditions are imposed on the tracer density $\phi$ such that $\phi = \SI{0}{\kilo\gram\per\meter\cubed}$ at the inlet boundary, and a zero normal gradient
$\partial \phi / \partial n = \SI{0}{\kilo\gram\per\meter\tothe{4}}$ is imposed at the outlet boundary.  There is no normal flow at the lower and upper boundaries.

The terrain is wave-shaped, specified by the surface elevation $h$ such that
\begin{subequations}
\begin{align}
   h(x) &= h^\star \cos^2 ( \alpha x )
%
\shortintertext{where}
%
   h^\star(x) &= \left\{ \begin{array}{l l}
       h_0 \cos^2 ( \beta x ) & \enskip \text{if $| x | < a$} \\
	0 & \enskip \text{otherwise}
    \end{array} \right.
\end{align}
\end{subequations}
where $a = \SI{25}{\kilo\meter}$ is the mountain envelope half-width, $h_0 = \SI{6}{\kilo\meter}$ is the maximum mountain height, $\lambda = \SI{8}{\kilo\meter}$ is the wavelength, \(\alpha = \pi / \lambda\) and \(\beta = \pi / (2a)\).  Note that, in order to make this test more challenging, the mountain height $h_0$ is double the mountain height used by \citet{schaer2002}.

A basic terrain-following (BTF) mesh is constructed by using the terrain profile to modify the uniform rectangular mesh.
The BTF method uses a linear decay function so that mesh layers become horizontal at the top of the model domain \citep{galchen-somerville1975a},
\begin{equation}
	z(x) = \left( H - h(x) \right) \left( z^\star / H \right) + h(x) \label{eqn:btf}
\end{equation}
where $z$ is the geometric height, $H$ is the height of the domain, $h(x)$ is the surface elevation and $z^\star$ is the computational height of a mesh layer.  If there were no terrain then $h = 0$ and $z = z^\star$.

A velocity field is prescribed with uniform horizontal flow aloft and zero flow near the ground,
\begin{align}
	u(z) = u_0 \left\{ \begin{array}{l l}
		1 & \enskip \text{if $z \geq z_2$} \\
		\sin^2 \left( \frac{\pi}{2} \frac{z - z_1}{z_2 - z_1} \right) & \enskip \text{if $z_1 < z < z_2$} \\
		0 & \enskip \text{otherwise}
	\end{array} \right.	
\end{align}
where $u_0 = \SI{10}{\meter\per\second}$, $z_1 = \SI{7}{\kilo\meter}$ and $z_2 = \SI{8}{\kilo\meter}$.
This results in a constant wind above $z_2$, and zero flow at \SI{7}{\kilo\meter} and below.

The discrete velocity field is defined using a streamfunction, \(\Psi\).  Given that \(u = -\partial \Psi / \partial z\), the streamfunction is found by vertical integration of the velocity profile:
\begin{align}
	\Psi(z) &= -\frac{u_0}{2} \left\{ \begin{array}{l l}
		\left( 2z - z_1 - z_2 \right) & \enskip \text{if $z > z_2$} \\
		z - z_1 - \frac{z_2 - z_1}{\pi} \sin \left(\pi \frac{z - z_1}{z_2-z_1}\right) & \enskip \text{if $z_1 < z \leq z_2$} \\
		0 & \enskip \text{if $z \leq z_1$}
	\end{array} \right.
\end{align}

A tracer with density $\phi$ is positioned upstream above the height of the terrain.  It has the shape
\begin{subequations}
\begin{align}
	\phi(x, z) &= \phi_0 \left\{ \begin{array}{l l}
		\cos^2 \left( \frac{\pi r}{2} \right) & \enskip \text{if $r \leq 1$} \\
		0 & \enskip \text{otherwise}
	\end{array} \right.
%
\intertext{with radius $r$ given by}
%
	r &= \sqrt{
		\left( \frac{x - x_0}{A_x} \right)^2 + 
		\left( \frac{z - z_0}{A_z} \right)^2
	}
\end{align}
%
\label{eqn:cubicFit:schaerAdvect:tracer}
\end{subequations}
where $A_x = \SI{25}{\kilo\meter}$, $A_z = \SI{3}{\kilo\meter}$ are the horizontal and vertical half-widths respectively, and $\phi_0 = \SI{1}{\kilogram\per\meter\cubed}$ is the maximum density of the tracer.  At $t = \SI{0}{\second}$, the tracer is centred at $(x_0, z_0) = (\SI{-50}{\kilo\meter}, \SI{12}{\kilo\meter})$ so that the tracer is upwind of the mountain, in the region of uniform flow above $z_2$.

Tests are integrated for \SI{10000}{\second} using s chosen for each mesh so that the maximum Courant number is about \num{0.4}.  This choice yields a time-step that is well below any stability limit, as recommended by \citet{lauritzen2012}.  By the end of integration the tracer is positioned downwind of the mountain.
The analytic solution at $t = \SI{10000}{\second}$ is centred at $(x_0, z_0) = (\SI{50}{\kilo\meter}, \SI{12}{\kilo\meter})$ with its shape unchanged from the initial condition.

\begin{figure}
	\centering
	\begin{subfigure}{\textwidth}
		\input{cubicFit/schaerAdvect/convergence}
		\phantomsubcaption\label{fig:cubicFit:schaerAdvect:convergence:l2}
		\phantomsubcaption\label{fig:cubicFit:schaerAdvect:convergence:linf}
		\phantomsubcaption\label{fig:cubicFit:tfAdvect:convergence:l2}
		\phantomsubcaption\label{fig:cubicFit:tfAdvect:convergence:linf}
	\end{subfigure}
%
	\caption{Numerical convergence of the two-dimensional tracer transport tests over mountains using
	(\subcaptionref{fig:cubicFit:schaerAdvect:convergence:l2}, \subcaptionref{fig:cubicFit:schaerAdvect:convergence:linf}) horizontal and
	(\subcaptionref{fig:cubicFit:tfAdvect:convergence:l2}, \subcaptionref{fig:cubicFit:tfAdvect:convergence:linf}) terrain-following velocity fields.
	$\ell_2$ errors (equation~\ref{eqn:l2-error}) and $\ell_\infty$ errors (equation~\ref{eqn:linf-error}) are marked at mesh spacings between \SI{5000}{\meter} and \SI{250}{\meter} using linearUpwind and cubicFit transport schemes on basic terrain-following and cut cell meshes.}
	\label{fig:cubicFit:schaerAdvect:convergence}
\end{figure}

To assess numerical convergence, a range of mesh spacings are chosen so that $\Delta x \mathbin{:} \Delta z = 2\mathbin{:}1$ to match the original test specification from \citet{schaer2002}.
Tests were performed using the linearUpwind and cubicFit schemes using BTF meshes and cut cell meshes with mesh spacings between $\Delta x = \SI{250}{\meter}$ and $\Delta x = \SI{5000}{\meter}$.
Error norms are calculated by subtracting the analytic solution from the numerical solution,
\begin{align}
	\ell_2 &= \sqrt{\frac{\sum_c \left(\phi - \phi_T \right)^2 \vol_c}{\sum_c \left(\phi_T^2 \vol_c \right)}} \label{eqn:l2-error} \\
	\ell_\infty &= \frac{\max_c \Mag{\phi - \phi_T}}{\max_c \Mag{\phi_T}} \label{eqn:linf-error}
\end{align}
where $\phi$ is the numerical value, $\phi_T$ is the analytic value, $\sum_c$ denotes a summation over all cells $c$ in the domain, and $\max_c$ denotes a maximum value of any cell.
The linearUpwind and cubicFit schemes are second-order convergent in the $\ell_2$ norm (figure~\ref{fig:cubicFit:schaerAdvect:convergence:l2}) and $\ell_\infty$ norm (figure~\ref{fig:cubicFit:schaerAdvect:convergence}) at all but the coarsest mesh spacings where errors are saturated for both schemes.

The cubicFit scheme achieves a given $\ell_2$ error using a mesh spacing that is almost twice as coarse as that needed by the linearUpwind scheme.  Doubling the mesh spacing results in a coarser mesh with four times fewer cells because the $\Delta x \mathbin{:} \Delta z$ aspect ratio is fixed.
Recall that the stencil for the cubicFit scheme has about twice as many cells as the stencil for the linearUpwind scheme.
Hence, for a given $\ell_2$ error, the computational cost during integration of the cubicFit scheme is about half the computational cost of the linearUpwind scheme.

\begin{figure}
	\centering
	\begin{subfigure}{\textwidth}
		\centering
		\includegraphics{thesis/cubicFit/schaerAdvect/fig-error.pdf}
		\phantomsubcaption\label{fig:cubicFit:schaerAdvect:error:btf:linearUpwind}
		\phantomsubcaption\label{fig:cubicFit:schaerAdvect:error:cutCell:linearUpwind}
		\phantomsubcaption\label{fig:cubicFit:schaerAdvect:error:btf:cubicFit}
		\phantomsubcaption\label{fig:cubicFit:schaerAdvect:error:cutCell:cubicFit}
		\phantomsubcaption\label{fig:cubicFit:tfAdvect:error:btf:linearUpwind}
		\phantomsubcaption\label{fig:cubicFit:tfAdvect:error:cutCell:linearUpwind}
		\phantomsubcaption\label{fig:cubicFit:tfAdvect:error:btf:cubicFit}
		\phantomsubcaption\label{fig:cubicFit:tfAdvect:error:cutCell:cubicFit}
	\end{subfigure}
	\caption{Tracer contours at the end of integration for the two-dimensional tracer transport tests over mountains using
	(\subcaptionref{fig:cubicFit:schaerAdvect:error:btf:linearUpwind},
	\subcaptionref{fig:cubicFit:schaerAdvect:error:cutCell:linearUpwind},
	\subcaptionref{fig:cubicFit:schaerAdvect:error:btf:cubicFit},
	\subcaptionref{fig:cubicFit:schaerAdvect:error:cutCell:cubicFit}) horizontal and 
	(\subcaptionref{fig:cubicFit:tfAdvect:error:btf:linearUpwind},
	\subcaptionref{fig:cubicFit:tfAdvect:error:cutCell:linearUpwind},
	\subcaptionref{fig:cubicFit:tfAdvect:error:btf:cubicFit},
	\subcaptionref{fig:cubicFit:tfAdvect:error:cutCell:cubicFit}) terrain-following velocity fields.  The numerical solution, marked with solid lines, and the analytic solution, marked with dashed lines, are plotted every \num{0.1}.  Tracer contours overlay a color error field, calculated by subtracting the analytic solution from the numerical solution.  Only the lowest \SI{20}{\kilo\meter} in the lee of the mountain is plotted.  The entire domain is \SI{300}{\kilo\meter} wide and \SI{25}{\kilo\meter} high.
	}
	
	\label{fig:cubicFit:schaerAdvect:error}
\end{figure}

Next, we examine the structure of numerical errors with test results using the linearUpwind and cubicFit transport schemes on BTF and cut cell meshes with $\Delta x = \SI{1000}{\meter}$ and $\Delta z = \SI{500}{\meter}$.
To obtain a maximum Courant number of about \num{0.4}, we choose $\Delta t = \inputval{schaerAdvect-cutCell-1000-linearUpwind/dt}$ on the cut cell mesh where the flow is aligned with mesh layers and there are no fluxes through upper and lower cell faces.
Since there is no flow below $z = \SI{7}{\kilo\meter}$, the time-step is not constrained by small, cut cells next to the lower boundary.
On the BTF mesh, $\Delta t$ is only \inputval{schaerAdvect-btf-1000-linearUpwind/dt} because the flow is misaligned with mesh layers, with fluxes through all four faces of cells above sloping terrain.

The highly-distorted BTF mesh presents a particular challenge to the linearUpwind scheme with the final numerical solution, marked by solid lines, losing its correct shape and maximum intensity compared to the analytic solution marked by dashed lines (figure~\ref{fig:cubicFit:schaerAdvect:error:btf:linearUpwind}).
The linearUpwind scheme produces a much better solution on the cut cell mesh, with only small phase errors apparent in figure~\ref{fig:cubicFit:schaerAdvect:error:cutCell:linearUpwind}.
Accuracy is much improved using the cubicFit scheme: on the BTF mesh, shape and maximum intensity are similar to the analytic solution (figure~\ref{fig:cubicFit:schaerAdvect:error:btf:cubicFit}) and, on the cut cell mesh, numerical errors are so small they are not visible (figure~\ref{fig:cubicFit:schaerAdvect:error:cutCell:cubicFit}).
The numerical and analytic contours overlay a color error field that reveals horizontal streaks of error on the BTF mesh (figure~\ref{fig:cubicFit:schaerAdvect:error:btf:linearUpwind},~\ref{fig:cubicFit:schaerAdvect:error:btf:cubicFit}) that were generated above the steepest mountain peaks before becoming trapped in the region of zero flow below $z = \SI{7}{\kilo\meter}$.

The horizontal transport test demonstrates that the cubicFit scheme is second-order convergent in the domain interior irrespective of mesh distortions.  Numerical errors are largest on terrain-following meshes, due either to misalignment of the flow with mesh layers, or to mesh distortions.
In the next section, we propose a new test in order to identify the primary cause of these numerical errors.
