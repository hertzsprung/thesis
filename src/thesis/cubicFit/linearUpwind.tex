\subsection{Multidimensional linear upwind transport scheme}

The multidimensional linear upwind scheme, called ``linearUpwind'' hereafter, is documented here since it provides a baseline accuracy for the experiments that follow.  The approximation of $\phi_F$ is calculated using a gradient reconstruction,
\begin{align}
	\phi_F &= \phi_u + \nabla_c\: \phi \cdot \left(\vect{x}_f - \vect{x}_c \right)
\end{align} 
where $\phi_u$ is the upwind value of $\phi$, and $\vect{x}_f$ and $\vect{x}_c$ are the position vectors of the face centroid and cell centroid respectively.
The gradient $\nabla_c \:\phi$ is calculated using Gauss's theorem:
\begin{align}
	\nabla_c\: \phi = \frac{1}{\vol_c} \sum_{f\in\:c} \widetilde{\phi}_F \Sf \label{eqn:linearUpwind-grad}
\end{align}
where $\widetilde{\phi}_F$ is linearly interpolated from the two neighbouring cells of face $f$.
The resulting stencil comprises all cells sharing a face with the upwind cell, including the upwind cell itself.  For a face in the interior of a two-dimensional rectangular mesh, the stencil for the linearUpwind scheme is a `$+$' shape with 5 cells.  On the same mesh, the stencil for the cubicFit scheme is more than twice the size with 12 cells.
For cells adjacent to boundaries having zero gradient boundary conditions, the boundary value is set to be equal to the cell centre value before equation~\eqref{eqn:linearUpwind-grad} is evaluated.
This implementation of the multidimensional linear upwind scheme is included with OpenFOAM \citep{openfoam-numerics}.
