\section{Transport schemes for arbitrary meshes}
\label{sec:cubicFit:transport}

The transport of a dependent variable $\phi$ in a prescribed, non-divergent velocity field $\vect{u}$ is given by the equation		
\begin{align}		
	\frac{\partial \phi}{\partial t} + \nabla \cdot \left( \vect{u} \phi \right) = 0 \text{ .} \label{eqn:advection}		
\end{align}
The time derivative is discretised using an explicit, two-stage, second-order Heun scheme,
\begin{subequations}
\begin{align}
	\phi^\star &= \phi^{(n)} + \Delta t \: g(\phi^{(n)}) \\
	\phi^{(n+1)} &= \phi^{(n)} + \frac{\Delta t}{2} \left[ g(\phi^{(n)}) + g(\phi^{\star}) \right]
\end{align} \label{eqn:heun}%
\end{subequations}
\unskip where \(g(\phi^{(n)}) = - \nabla \cdot (\vect{u} \phi^{(n)})\) at time level \(n\).
This two-stage second-order time-stepping scheme is similar to the three-stage second-order time-stepping scheme used later in a model of the fully compressible Euler equations (section~\ref{sec:slanted:exnerFoamH}), which uses an additional time-stage to converge upon the semi-implicit solution.
The two-stage second-order time-stepping scheme is used for both the cubicFit scheme and the multidimensional linear upwind scheme.
Although the Heun scheme is unstable for a linear oscillator \citep{durran2013} and for solving the transport equation using centred, linear differencing, it is stable when it is used for transport schemes with sufficient upwinding \citep[p. 149]{hundsdorfer-verwer2013}.

Using the finite volume method, the velocity field is prescribed at face centroids and the dependent variable is stored at cell centroids.  The divergence term in equation~\eqref{eqn:advection} is discretised using Gauss's theorem,
\begin{align}
	\nabla \cdot \left( \vect{u} \phi \right) \approx \frac{1}{\vol_c} \sum_{f \in\:c} \vect{u}_f \cdot \Sf \phi_F \label{eqn:gauss-div}
\end{align}
where subscript $f$ denotes a value stored at a face and subscript $F$ denotes a value approximated at a face from surrounding values.  $\vol_c$ is the cell volume, $\vect{u}_f$ is a velocity vector prescribed at a face, $\Sf$ is the surface area vector with a direction outward normal to the face and a magnitude equal to the face area, $\phi_F$ is an approximation of the dependent variable at the face, and $\sum_{f \in\:c}$ denotes a summation over all faces $f$ bordering cell $c$.

This discretisation is applicable to arbitrary meshes.  A necessary condition for stability is given by the multidimensional Courant number \citep{weller-shahrokhi2014},
\begin{align}
	\mathrm{Co}_c = \frac{\Delta t}{2 \mathcal{V}_c} \sum_{f \in\: c} \Mag{\vect{u} \cdot \Sf} \label{eqn:co}
\end{align}
such that, for all cells $c$ in the domain, $\mathrm{Co}_c$ is less than or equal to some constant that depends upon the spatial and temporal discretisation.  Hence, stability is constrained by the maximum Courant number of any cell in the domain.

The accurate approximation of the dependent variable at the face, $\phi_F$, is key to the overall accuracy of the transport scheme. The cubicFit scheme and the multidimensional linear upwind scheme differ in their approximations, and these approximation methods are described next.

