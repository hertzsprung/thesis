\section{Transport in a terrain-following velocity field}
\label{sec:cubicFit:tfAdvect}

In the horizontal transport test, results were least accurate on the BTF mesh where the mesh was most distorted and flow was misaligned with mesh layers.
Here, we formulate a new tracer transport test in which the velocity field is everywhere tangential to the basic terrain-following mesh layers.
The flow is then aligned with the BTF mesh layers, but the points in the linearUpwind and cubicFit stencils are not uniformly distributed because the BTF mesh is distorted.
Conversely, the flow is misaligned with the cut cell mesh layers but, except in cut cells next to the ground, the cut cell mesh is undistorted.
This test determines whether the primary source of numerical error is due to mesh distortions or misalignment of the flow with mesh layers.

The domain size, mountain profile, initial tracer profile and boundary conditions are the same as those in the horizontal tracer advection test in section~\ref{sec:cubicFit:schaerAdvect}.
The discrete velocity field is calculated using a streamfunction $\Psi$ in the same way as the horizontal transport test.
Here, we define a different streamfunction that yields a velocity field that follows the BTF mesh layers given by equation~\eqref{eqn:btf} such that
\begin{equation}
	\Psi(x,z) = -u_0 H_1 \frac{z - h}{H_1 - h} \label{eqn:streamfunc-btf}
\end{equation}
where $u_0 = \SI{10}{\meter\per\second}$, which is the horizontal velocity where $h(x) = 0$.
The velocity field follows the lower boundary and becomes entirely horizontal at $H_1 = H = \SI{25}{\kilo\meter}$, hence, there is no normal flow at the lower and upper boundaries.
In the domain interior, the flow is predominantly horizontal, with non-zero vertical velocities only above sloping terrain.

The horizontal and vertical components of velocity, $u$ and $w$, are given by
\begin{align}
	u &= -\frac{\partial \Psi}{\partial z} = u_0 \frac{H_1}{H_1 - h}, \quad w = \frac{\partial \Psi}{\partial x} = u_0 H_1 \frac{\mathrm{d} h}{\mathrm{d} x} \frac{H_1 - z}{\left( H_1 - h \right)^2} \label{eqn:slanted:uw-btf} \text{ ,}\\
	\frac{\mathrm{d} h}{\mathrm{d} x} &= - h_0 \left[ 
		\beta \cos^2 \left( \alpha x \right) \sin \left( 2 \beta x \right) +
		\alpha \cos^2 \left( \beta x \right) \sin \left( 2 \alpha x \right)
	\right] \text{ .}
\end{align}
Unlike the horizontal transport test, the velocity field presented here extends from the top of the domain all the way to the ground.

An analytic solution at \SI{10000}{\second} is obtained by calculating the new horizontal position of the tracer.  Integrating along the trajectory yields $t$, the time taken to move from the left side of the mountain at $-a$, to the right side of the mountain at $a$,
\begin{align}
	\mathrm{d}t &= \mathrm{d}x / u(x) \\
	t &= \int_{-a}^a \frac{H_1 - h(x)}{u_0 H_1}\:\mathrm{d}x \\
	t &= \frac{2a}{u_0} - \frac{h_0}{16 u_0 H_1} \left[ 4x + \frac{\sin 2 (\alpha + \beta) x}{\alpha + \beta} +
	\frac{\sin 2(\alpha - \beta) x}{\alpha - \beta} + 2 \left( \frac{\sin 2\alpha x}{\alpha} + \frac{\sin 2\beta x}{\beta} \right) \right]_{-a}^a \label{eqn:cubicFit:tfAdvect:trajectory}
\end{align}
Because the velocity field is non-divergent, the flow accelerates over mountain ridges and the tracer travels \SI{2997.162}{\meter} further compared to the tracer in a purely horizontal velocity field.  The vertical tracer position is unchanged downwind of the mountains because flow is parallel to the mesh layers.

To enable comparisons with the horizontal transport test, results are obtained using the linearUpwind and cubicFit transport schemes on BTF and cut cell meshes with $\Delta x = \SI{1000}{\meter}$ and $\Delta z = \SI{500}{\meter}$.
To obtain a maximum Courant number of about \num{0.4}, we choose $\Delta t = \inputval{tfAdvect-btf-1000-linearUpwind/dt}$ on the BTF mesh where flow is aligned with mesh layers.
The cut cell mesh suffers from the small cell problem, having a more stringent time-step constraint of $\Delta t = \inputval{tfAdvect-cutCell-1000-linearUpwind/dt}$.  Recall that, in this test, there is flow everywhere in the domain, and it is flow through arbitrarily small cut cells that imposes the more stringent time-step constraint.

\begin{figure}
	\centering
	\includegraphics{thesis/cubicFit/tfAdvect/fig-tracer.pdf}
	\caption{\TODO{make plot as big as possible so dashed contours are easier to make out}
	Tracer contours transported above mountains in a terrain-following velocity field at $t = \SI{0}{\second}$, \SI{5000}{\second}, and \SI{10000}{\second} using the cubicFit transport scheme on a BTF mesh.
	The analytic solution at $t = \SI{10000}{\second}$ is plotted with dashed contours.
	All contour intervals are \num{0.1}.  The terrain profile is shown immediately above the $x$ axis.
	The region highlighted in orange marks the region plotted in the panels of figure~\ref{fig:cubicFit:schaerAdvect:error}.
	Only the central \SI{200}{\kilo\meter} of the domain is shown.  The entire domain is \SI{300}{\kilo\meter} wide and \SI{25}{\kilo\meter} high.}
	\label{fig:cubicFit:tfAdvect:tracer}
\end{figure}

Figure~\ref{fig:cubicFit:tfAdvect:tracer} shows results using the cubicFit scheme on the BTF mesh, illustrating the evolution of the tracer with snapshots plotted every \SI{5000}{\second}.  At $t = \SI{5000}{\second}$, the tracer is distorted by the terrain-following velocity field but, by $t = \SI{10000}{\second}$, the tracer has correctly returned to its original shape, with some phase errors apparent when comparing the numerical solution (solid contours) with the analytic solution (dashed contours).  The region highlighted in orange corresponds to the region plotted in figure~\ref{fig:cubicFit:schaerAdvect:error}, where tracer contours and numerical errors are plotted at $t = \SI{10000}{\second}$.

Unlike the horizontal transport test, results are most accurate on the BTF mesh (linearUpwind, figure~\ref{fig:cubicFit:tfAdvect:error:btf:linearUpwind}; cubicFit, figure~\ref{fig:cubicFit:tfAdvect:error:btf:cubicFit}) and least accurate on the cut cell mesh
(linearUpwind, figure~\ref{fig:cubicFit:tfAdvect:error:cutCell:linearUpwind}; cubicFit, figure~\ref{fig:cubicFit:tfAdvect:error:cutCell:cubicFit}).
Hence, we conclude that the accuracy of the transport schemes depends upon alignment of the flow with mesh layers, and accuracy is mostly unaffected by mesh distortions.
The error structures on the cut cell mesh in this test (\ref{fig:cubicFit:tfAdvect:error:cutCell:linearUpwind}, \ref{fig:cubicFit:tfAdvect:error:cutCell:cubicFit}) are similar to the error structures on the BTF mesh in the horizontal transport test (\ref{fig:cubicFit:schaerAdvect:error:btf:linearUpwind}, \ref{fig:cubicFit:schaerAdvect:error:btf:cubicFit}), and the phase error using the linearUpwind scheme on the BTF mesh (\ref{fig:cubicFit:tfAdvect:error:btf:linearUpwind}) closely resembles the error on the cut cell mesh in the horizontal transport test (\ref{fig:cubicFit:schaerAdvect:error:cutCell:linearUpwind}).

Perhaps surprisingly, errors are slightly larger using the cubicFit scheme on the BTF mesh (\ref{fig:cubicFit:tfAdvect:error:btf:cubicFit}) compared to those obtained using the linearUpwind scheme (\ref{fig:cubicFit:tfAdvect:error:btf:linearUpwind}).
At finer mesh spacings, however, cubicFit is more accurate on BTF and cut cell meshes in both the $\ell_2$ norm (figure~\ref{fig:cubicFit:tfAdvect:convergence:l2}) and $\ell_\infty$ norm (figure~\ref{fig:cubicFit:tfAdvect:convergence:linf}).  Once again, both transport schemes are second-order convergent irrespective of mesh distortions or misalignment of the flow with mesh layers.

In both horizontal and terrain-following transport tests, which are both variations on the standard test case by \citet{schaer2002}, the linearUpwind and cubicFit transport schemes are second-order convergent irrespective of mesh distortions or misalignment of the flow with mesh layers.
Together, the horizontal and terrain-following transport tests demonstrate that numerical accuracy depends primarily on the alignment of the flow with mesh layers.
