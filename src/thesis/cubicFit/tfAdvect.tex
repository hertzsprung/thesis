\section{Transport in a terrain-following velocity field}

\TODO{
\begin{itemize}
	\item Conclusion: Misalignment of the velocity field with mesh layers is the primary source of numerical error
	\item Conclusion: 2nd-order convergence again
\end{itemize}
}

In the horizontal transport test, results were least accurate on the BTF mesh where the mesh was most distorted and flow was misaligned with the mesh layers.
Here, we formulate a new tracer transport test in which the velocity field is everywhere tangential to the basic terrain-following mesh layers.
The flow is then aligned with the BTF mesh layers, but the points in the linearUpwind and cubicFit stencils are not uniformly distributed because the BTF mesh is distorted.
Conversely, the flow is misaligned with the cut cell mesh layers but, except in cut cells next to the ground, the cut cell mesh is undistorted.
This test determines whether the primary source of numerical error is due to mesh distortions or misalignment of the flow with mesh layers.

The spatial domain, mountain profile, initial tracer profile and discretisation are the same as those in the horizontal tracer advection test, \TODO{except for the choice of timesteps... what are the timesteps?}
The discrete velocity field is calculated using a streamfunction in the same way as the horizontal transport test.
Here, we define a streamfunction $\Psi$ is that provides a velocity field that follows the BTF mesh layers given by equation~\eqref{eqn:btf} such that
\begin{equation}
	\Psi(x,z) = -u_0 H_1 \frac{z - h}{H_1 - h} \label{eqn:streamfunc-btf}
\end{equation}
where $u_0 = \SI{10}{\meter\per\second}$, which is the horizontal velocity where $h(x) = 0$.
The velocity field follows the lower boundary and becomes entirely horizontal at $H_1 = H = \SI{25}{\kilo\meter}$, hence, there is no normal flow at the lower and upper boundaries.
In the domain interior, the flow is predominantly horizontal, with non-zero vertical velocities only above sloping terrain.

The horizontal and vertical components of velocity, $u$ and $w$, are given by
\begin{align}
	u &= -\frac{\partial \Psi}{\partial z} = u_0 \frac{H_1}{H_1 - h}, \quad w = \frac{\partial \Psi}{\partial x} = u_0 H_1 \frac{\mathrm{d} h}{\mathrm{d} x} \frac{H_1 - z}{\left( H_1 - h \right)^2} \label{eqn:slanted:uw-btf} \text{ ,}\\
	\frac{\mathrm{d} h}{\mathrm{d} x} &= - h_0 \left[ 
		\beta \cos^2 \left( \alpha x \right) \sin \left( 2 \beta x \right) +
		\alpha \cos^2 \left( \beta x \right) \sin \left( 2 \alpha x \right)
	\right] \text{ .}
\end{align}
Unlike the horizontal transport test, the velocity field presented here extends from the top of the domain all the way to the ground.

An analytic solution at \SI{10000}{\second} is obtained by calculating the new horizontal position of the tracer.  Integrating along the trajectory yields $t$, the time taken to move from the left side of the mountain at $-a$, to the right side of the mountain at $a$,
\begin{align}
	\mathrm{d}t &= \mathrm{d}x / u(x) \\
	t &= \int_{-a}^a \frac{H_1 - h(x)}{u_0 H_1}\:\mathrm{d}x \\
	t &= \left. \frac{2a}{u_0} - \frac{h_0}{16 u_0 H_1} \left[ 4x + \frac{\sin 2 (\alpha + \beta) x}{\alpha + \beta} +
\frac{\sin 2(\alpha - \beta) x}{\alpha - \beta} + 2 \left( \frac{\sin 2\alpha x}{\alpha} + \frac{\sin 2\beta x}{\beta} \right) \right] \right\rvert_{-a}^a
\end{align}
Because the velocity field is non-divergent, the flow accelerates over mountain ridges and the tracer travels \TODO{\SI{123}{\meter}} further compared to the tracer in a purely horizontal velocity field.  Tracer height is unchanged downwind of the mountains because flow is parallel to the mesh layers.

\begin{figure}
	\centering
	\includegraphics{thesis/cubicFit/tfAdvect/fig-tracer.pdf}
	\caption{\TODO{tracer evolution with BTF cubicFit}}
	\label{fig:cubicFit:tfAdvect:tracer}
\end{figure}
