\section{Deformational flow on a sphere}
\label{sec:cubicFit:deformationSphere}

\TODO{The tests so far have used flows that are mostly uniform on meshes that are based on rectangular cells.}
To ensure that the cubicFit transport scheme is suitable for complex flows on a variety of meshes, we use a standard test of deformational flow on a spherical Earth, as specified by \citet{lauritzen2012}.  
Results are compared between linearUpwind and cubicFit schemes using hexagonal-icosahedral meshes and cubed-sphere meshes.
Hexagonal-icosahedral meshes are constructed by successive refinement of a regular icosahedron following the approach by \citet{thuburn2014,heikes-randall1995a,heikes-randall1995b} without any mesh twisting.
Cubed-sphere meshes are constructed using an equi-distant gnomic projection of a cube having a uniform Cartesian mesh on each panel \citep{staniforth-thuburn2012}.

Following appendix A9 in \citet{lauritzen2014}, the average equatorial spacing $\Delta \lambda$ is used as a measure of mesh spacing.  It is defined as
\begin{align}
	\Delta \lambda = \ang{360} \frac{\overline{\Delta x}}{2 \pi R_e}
\end{align}
where $\overline{\Delta x}$ is the mean distance between cell centres and $R_e = \SI{6.3712e6}{\meter}$ is the radius of the Earth.

The deformational flow test specified by \citet{lauritzen2012} comprised six elements:
\begin{enumerate}
\item a convergence test using a Gaussian-shaped tracer
\item a ``minimal'' resolution test using a cosine bell-shaped tracer
\item a test of filament preservation
\item a test using a ``rough'' slotted cylinder tracer
\item a test of correlation preservation between two tracers
\item a test using a divergent velocity field
\end{enumerate}
We assess the cubicFit scheme using the first two tests only.  We do not consider filament preservation, correlation preservation, or the transport of a ``rough'' slotted cylinder because no shape-preserving filter has yet been developed for the cubicFit scheme.  Stable results were obtained when testing the cubicFit scheme using a divergent velocity field, but no further analysis is made here.

The first deformational flow test uses an infinitely continuous initial tracer that is transported in a non-divergent, time-varying, rotational velocity field.
The velocity field deforms two Gaussian `hills' of tracer into thin vortical filaments.  Half-way through the integration the rotation reverses so that the filaments become circular hills once again.  The analytic solution at the end of integration is identical to the initial condition.
A rotational flow is superimposed on a time-invariant background flow in order to avoid error cancellation.
The non-divergent velocity field is defined by the streamfunction $\Psi$,
\begin{align}
	\Psi(\lambda, \theta, t) = \frac{10 R_e}{T} \sin^2 \left(\lambda'\right) \cos^2 \left(\theta\right) \cos \left( \frac{\pi t}{T} \right) - \frac{2 \pi R_e}{T} \sin\left(\theta\right)
\end{align}
where $\lambda$ is a longitude, $\theta$ is a latitude, $\lambda' = \lambda - 2 \pi t / T$, and $T = \num{12}$ days is the duration of integration.  The time-step is chosen such that the maximum Courant number is about 0.4.

The initial tracer density $\phi$ is defined as the sum of two Gaussian hills,
\begin{align}
	\phi = \phi_1(\lambda, \theta) + \phi_2(\lambda, \theta) \text{ .}
\end{align}
An individual hill $\phi_i$ is given by
\begin{align}
	\phi_i(\lambda, \theta) = \phi_0 \exp\left( -b \left( \frac{\Mag{\vect{x} - \vect{x}_i}}{R_e} \right)^2 \right)
\end{align}
where $\phi_0 = \SI{0.95}{\kilo\gram\per\meter\cubed}$ and $b = 5$.  The Cartesian position vector $\vect{x} = (x,y,z)$ is related to the spherical coordinates $(\lambda, \theta)$ by
\begin{align}
	(x,y,z) = (R_e \cos \theta \cos \lambda, R_e \cos \theta \sin \lambda, R_e \sin \theta) \label{eqn:cubicFit:spherical-cartesian} \text{ .}
\end{align}
The centre of hill $i$ is positioned at $\vect{x}_i$.  In spherical coordinates, two hills are centred at
\begin{align}
	(\lambda_1,\theta_1) &= (5 \pi /6, 0) \\
	(\lambda_2,\theta_2) &= (7 \pi /6, 0)
\end{align}

\begin{figure}
	\centering
%	\includegraphics{../fig-deformationSphere-initialTracer/fig-deformationSphere-initialTracer.pdf}
	\caption{Tracer fields for the deformational flow test using initial Gaussian hills.  The tracer is deformed by the velocity field before the rotation reverses to return the tracer to its original distribution: (a) the initial tracer distribution at $t = \SI{0}{\second}$; (b) by $t=T/2$ the Gaussian hills are stretched into a thin S-shaped filament; (c) at $t=T$ the tracer resembles the initial Gaussian hills except for some distortion and diffusion due to numerical errors.  Results were obtained with the cubicFit scheme on a hexagonal-icosahedral mesh with an average equatorial mesh spacing of $\Delta \lambda = \ang{0.542}$.}
	\label{fig:cubicFit:deformationSphere-evolution}
\end{figure}

The results in figure~\ref{fig:cubicFit:deformationSphere-evolution} are obtained using the cubicFit scheme on a hexagonal-icosahedral mesh with $\Delta \lambda = \ang{0.542}$.
The initial Gaussian hills are shown in figure~\ref{fig:cubicFit:deformationSphere-evolution}a.  At $t=T/2$ the tracer has been deformed into an S-shaped filament (figure~\ref{fig:cubicFit:deformationSphere-evolution}b).  By $t=T$ the tracer has almost returned to its original distribution except for some slight distortion and diffusion that are the result of numerical errors (figure~\ref{fig:cubicFit:deformationSphere-evolution}c).

\begin{figure}
	\centering
	\input{cubicFit/deformationSphere-gaussiansConvergence}
	\caption{Numerical convergence of the deformational flow test on the sphere using initial Gaussian hills.  $\ell_2$ errors (\TODO{equation~\ref{eqn:l2-error}}) and $\ell_\infty$ errors (\TODO{equation~\ref{eqn:linf-error}}) are marked at mesh spacings between \ang{8.61} and \ang{0.271} using the linearUpwind scheme (dotted lines) and the cubicFit scheme (solid lines) on hexagonal-icosahedral meshes and cubed-sphere meshes.}
	\label{fig:cubicFit:deformationSphere-gaussian-convergence}
\end{figure}

To determine the order of convergence and relative accuracy of the linearUpwind and cubicFit schemes, the same test was performed at a variety of mesh spacings betweeen $\Delta \lambda = \ang{8.61}$ and $\Delta \lambda = \ang{0.271}$ on hexagonal-icosahedral meshes and cubed-sphere meshes.  The results are shown in figure~\ref{fig:cubicFit:deformationSphere-gaussian-convergence}.
The solution is slow to converge at coarse resolutions, and this behaviour agrees with the results from \citet{lauritzen2012}.  Both linearUpwind and cubicFit schemes achieve second-order accuracy at smaller mesh spacings. 
For any given mesh type and mesh spacing, the cubicFit scheme is more accurate than the linearUpwind scheme.
Results are more accurate using hexagonal-icosahedral meshes compared to cubed-sphere meshes.  It is not known whether the larger errors on cubed-sphere meshes are due to mesh non-uniformities at panel corners but there is no evidence of grid imprinting in the error fields (not shown).

A slightly more challenging variant of the same test is performed using a quasi-smooth tracer field defined as the sum of two cosine bells,
\begin{align}
	\phi =
	\begin{cases}
		b + c \phi_1(\lambda, \theta) & \quad \text{if $r_1 < r$,} \\
		b + c \phi_2(\lambda, \theta) & \quad \text{if $r_2 < r$,} \\
		b			      & \quad \text{otherwise.}
	\end{cases}
\end{align}
The velocity field is the same as before.  This test is used to determine the ``minimal'' resolution, $\Delta \lambda_m$, which is specified by \citet{lauritzen2012} as the coarsest mesh spacing for which $\ell_2 \approx 0.033$.

\begin{table}
	\robustify\itshape
	\centering
	\begin{tabular}{l l S[detect-weight, detect-shape, detect-mode]}
\toprule
	Transport scheme & Mesh type & {Minimal resolution (\si{\degree})} \\
\midrule
	linearUpwind & Cubed-sphere & \itshape 0.15 \\
	\makecell{FARSIGHT, grid-point semi-Lagrangian \\ \quad\citep{white-dongarra2011}} & Cubed-sphere & 0.1875 \\
	linearUpwind & Hexagonal-icosahedral & \itshape 0.2 \\
	\makecell{SLFV-SL, swept-area scheme \\ \quad\citep{miura2007}} & Hexagonal-icosahedral & 0.25 \\
	cubicFit & Cubed-sphere & \itshape 0.25 \\
	cubicFit & Hexagonal-icosahedral & 0.3 \\
	\makecell{ICON-FFSL, swept-area scheme \\ \quad\citep{miura2007}} & Triangular-icosahedral & 0.42 \\
\bottomrule
\end{tabular}
%
	\caption{Minimal resolutions for the cubicFit and linearUpwind schemes in the test of deformational flow using cosine bells.  Italicised values have been extrapolated using the second-order convergence obtained at coarser mesh spacings.  For comparison with existing models, some results are also included for unlimited versions of the transport schemes from the intercomparison by \citet{lauritzen2014}.}
	\label{tab:deformationSphere:minimal-resolution}
\end{table}

The minimal resolution for the cubicFit scheme on a hexagonal-icosahedral mesh is about $\Delta \lambda_m = \ang{0.3}$.  Tests were not performed at mesh spacings finer than $\Delta \lambda = \ang{0.271}$ but approximate minimal resolutions have been extrapolated from the second-order convergence that is found at fine mesh spacings.  These minimal resolutions are presented in table~\ref{tab:deformationSphere:minimal-resolution} along with a selection of transport schemes having similar minimal resolutions from the model intercomparison by \citet{lauritzen2014}.

The series of deformational flow tests presented here demonstrate that the cubicFit scheme is suitable for transport on spherical meshes based on quadrilaterals and hexagons.  The cubicFit scheme is largely insensitive to the mesh type, and results are more accurate compared to the linearUpwind scheme for a given mesh type and mesh spacing.  Neither scheme requires special treatment at the corners of cubed-sphere panels.

