\chapter{High-order transport for arbitrary meshes}
\label{ch:highOrder}

\TODO{
Motivation
\begin{itemize}	
	\item cubicFit is suitable for flows over steep terrain but it is only second-order convergent despite having a large stencil and a cubic/quadratic polynomial
	\item high-order schemes on coarser meshes can be more computationally efficient than lower-order schemes on finer meshes... sure I saw a citation for this recently
	\item existing high-order FV schemes exist, but are they more computationally expensive? or unsuitable for arbitrary meshes? identify applications of high-order FV schemes to atmospheric models
	\item we want to retain the low computational cost of cubicFit but achieve high-order convergence at the same time
\end{itemize}
}
\TODO{
\begin{itemize}	
\item test: solid body rotation on 2D plane, uniform and distorted meshes following \citet{leonard1996,chen2017} (will need ghost cells for periodic BCs), \textbf{OR}
\item test: solid body rotations on spherical shell, cubed-sphere and hexagonal icosahedra \citep{lauritzen2012}
\item test: horizontal flow above a mountain on a basic terrain-following mesh \citep{schaer2002,shaw2017} -- the cosBell tracer field might be insufficiently smooth for high-order but still a useful comparison with cubicFit, and returns to the mountainous theme!
\end{itemize}
}
\TODO{somewhere in chapter intro: We define \textit{high-order convergence} as any rate of convergence beyond second-order.}

The cubicFit transport scheme acheives only second-order convergence even though it includes high-order polynomial terms.  The cubicFit scheme uses a sub-grid reconstruction that fits a polynomial over known values stored at cell centre points, and it is this point-wise approach that limits the scheme to second-order convergence.
\citet{devendran2017} developed a high-order finite volume scheme for solving Poisson’s equation, and they achieved high-order convergence by constraining the polynomial fit so that the average of the polynomial integrated over a cell volume equals the cell average value.
We apply their approach to obtain a transport scheme with high-order convergence.
Since it has much in common with the cubicFit transport scheme, we name this high-order transport scheme `highOrderFit'.

\section{High-order finite volume formulation}
\label{sec:highOrderFit:scheme}

Integrating the flux-form transport equation \eqref{eqn:advection} over a volume $\vol$ and using Gauss's divergence theorem,
\begin{align}
	\int_\vol \frac{\partial \phi}{\partial t} d\vol = - \int_{\partial \vol} \phi \vect{u} \cdot \unitn dA \label{eqn:highOrder:integrated-advection}
\end{align}
where $\unitn$ is the outward unit normal vector.
Using the method of lines, the time derivative is discretised using the classical fourth-order Runge–Kutta time-stepping scheme \citep[p. 53]{durran2013}, and the spatial discretisation is described next.
For a polygonal cell with faces $f$ equation~\eqref{eqn:highOrder:integrated-advection} becomes
\begin{align}
	\int_\vol \frac{\partial \phi}{\partial t} d\vol = - \sum_f \int_{\area_f} \phi \vect{u} \cdot \unitn d\area_f \label{eqn:highOrder:advection}
\end{align}
where $\area_f$ is the area of face $f$.
If $\phi$ is a sufficiently smooth field then it can be approximated to $P$-order accuracy by replacing $\phi$ with a polynomial interpolant $\psi$,
\begin{align}
	\psi = \sum_{\Mag{\vect{p}} \leq P} a_{\vect{p}} \left(\vect{x} - \vect{x}_0 \right)^\vect{p} \label{eqn:highOrder:interpolant}
\end{align}
where $a_\vect{p}$ are unknown polynomial coefficients, $\vect{x}_0$ is a fixed position, and $P$ is the total polynomial order.
Note that we use the multi-index notation such that $\Mag{\vect{p}} = p_1 + \ldots + p_n$ and
\begin{align}
	a_{\vect{p}} \left( \vect{x} - \vect{x}_0 \right)^\vect{p} = a_\vect{p} \prod_{d=1}^D \left( x_d - x_{0_d} \right)^{p_d} \text{ .}
\end{align}
where $D$ is the number of physical dimensions.
As an example, the exponents $\vect{p}$ in two dimensions $(x, y)$ with $\Mag{\vect{p}} \leq 1$ are $(0, 0)$, $(1, 0)$ and $(0, 1)$, hence the two-dimensional polynomial interpolant for a total polynomial order $P = 1$ is
\begin{align}
	\psi = a_{0,0} + a_{1,0} \left( x - x_0 \right) + a_{0,1} \left( y - y_0 \right) \text{ .}
\end{align}
Replacing $\phi$ in \eqref{eqn:highOrder:advection} with $\psi$ in \eqref{eqn:highOrder:interpolant} we obtain an expression for the face flux,
\begin{align}
	\int_\area \phi \vect{u} \cdot \unitn d\area = \uf \cdot \unitn \sum_{\Mag{\vect{p}} \leq P} a_{\vect{p}} \moment_\area^\vect{p} \label{eqn:highOrder:face-flux}
\end{align}
where $\moment_\area^\vect{p} = \int_\area \left( \vect{x} - \vect{x}_0 \right)^\vect{p} d \area$ is the $\vect{p}$-th moment of area $\area$.
\TODO{here we're assuming that $\vect{u}$ is smoother than $\phi$, see \citet{methven-hoskins1999} for a justification?}
Therefore, the face flux can be calculated by finding the the polynomial coefficients $a_\vect{p}$.

Following the same approach as the cubicFit transport scheme, taking a total polynomial order $P = 3$ gives 9 polynomial terms with polynomial coefficients calculated using the same upwind-biased stencil.
For every cell in the stencil we require that the average of the polynomial integrated over a cell volume equals the cell average value,
\begin{align}
	\volave{\psi} = \volave{\phi} \label{eqn:highOrder:equal-volumes}
%
\intertext{where the average over volume $\vol$ is}
%
	\volave{\psi} = \frac{1}{\vol} \int_\vol \psi d\vol \text{ .} \label{eqn:highOrder:volume-average}
\end{align}
Using equations~\eqref{eqn:highOrder:interpolant} and \eqref{eqn:highOrder:volume-average} we can rewrite equation~\eqref{eqn:highOrder:equal-volumes} as
\begin{align}
	\frac{1}{\moment_\vol^\vect{0}} \sum_{\Mag{\vect{p}} \leq P} a_\vect{p} \moment_\vol^\vect{p} = \volave{\phi}
\end{align}
where $\moment_\vol^\vect{p} = \int_\vol \left( \vect{x} - \vect{x}_0 \right)^\vect{p} d\vol$ is the $\vect{p}$-th moment of volume $\vol$, and the zeroth moment $\moment_\vol^\vect{0}$ is the volume.
For $m$ polynomial terms and a stencil with $n$ cells, we calculate a face flux by choosing $\vect{x}_0$ to be the position of the face centre, then we write the linear system
\begin{align}
	\begin{bmatrix}
		\moment_{\vol_1}^{\vect{p}_1}/\moment_{\vol_1}^\vect{0} & \cdots & \moment_{\vol_1}^{\vect{p}_m}/\moment_{\vol_1}^\vect{0} \\
		\vdots & & \vdots \\
		\moment_{\vol_n}^{\vect{p}_1}/\moment_{\vol_n}^\vect{0} & \cdots & \moment_{\vol_n}^{\vect{p}_m}/\moment_{\vol_n}^\vect{0}
	\end{bmatrix}
	\begin{bmatrix}
		a_{\vect{p}_1} \\
		\vdots \\
		a_{\vect{p}_m}
	\end{bmatrix}
	=
	\begin{bmatrix}
		\langle \phi \rangle_{\vol_1} \\
		\vdots \\
		\langle \phi \rangle_{\vol_n}
	\end{bmatrix} \label{eqn:highOrder:moment-matrix}
\end{align}
which can be written as
\begin{align}
	\vect{B} \vect{a} = \bm{\phi} \text{ .}
\end{align}
where $\vect{B}$ is the stencil matrix, which is constructed using only the mesh geometry.
The highOrderFit scheme generates stencils using the same procedure as the cubicFit scheme.
Assuming a stencil comprises at least as many cells as there are polynomial coefficients then $n \geq m$ and the matrix equation can be solved using a least-squares approach to find the unknown coefficients $\vect{a}$.

To obtain a stable transport scheme, we follow the approach of the cubicFit scheme by introducing multipliers $\vect{m}$ to obtain
\begin{align}
	\vect{\tilde{B}} \vect{a} = \vect{m} \cdot \bm{\phi}
\end{align}
where $\vect{\tilde{B}} = \vect{M} \vect{B}$ and $\vect{M} = \mathrm{diag}(\vect{m})$.
The upwind cell and downwind cell have multipliers $m_u = 2^{10}$ and $m_d = 2^{10}$ respectively, and all peripheral points have multipliers $m_p = 1$.

The calculation of high-order cell volume moments and surface moments are required by equations~\eqref{eqn:highOrder:moment-matrix} and~\eqref{eqn:highOrder:face-flux} respectively.  These volume and surface moments can be calculated exactly using the method of \citet{tuzikov2003}.
We follow their method but, in order to avoid any degenerate triangles, we introduce a centre point shared by all triangles instead of triangulating polygons with only existing vertices.

While the highOrderFit transport scheme uses a total polynomial order $P = 3$ for stencils in the domain interior, a total polynomial order $P = 1$ is used for stencils near the boundary having fewer than 12 cells.
This reduction in total polynomial order ensures that matrix equations are never underconstrained.
This thesis does not assess the accuracy of the highOrderFit scheme near boundaries, and so the more sophisticated boundary treatment implemented in the cubicFit scheme has not been implemented in the highOrderFit scheme.


\section{Deformational flow on a plane}
\label{sec:highOrderFit:deformationPlane}

The standard test case by \citet{lauritzen2012} of deformational flow on a two-dimensional spherical shell was adapted by \citet{chen2017} for use on a two-dimensional Cartesian plane.
Since the highOrderFit formulation described in section~\ref{sec:highOrderFit:scheme} has not been extended to spherical geometry, we use the test case by \citet{chen2017} to measure the order of convergence of the highOrderFit transport scheme in a time-varying, rotational velocity field.
Tests are performed on uniform meshes, and meshes with distortions similar to those found on the cubed-sphere.

Following \citet{chen2017}, the domain is defined on a rectangular $x$--$y$ plane that is $2\pi$ wide and $\pi$ tall.  The domain is periodic in the $x$ direction with no normal flow imposed at the upper and lower boundaries.
The discrete velocity field is defined using the streamfunction,
\begin{align}
	\Psi = \frac{\hat{\Psi}}{T} \sin^2 \left( 2 \pi \left( \frac{x}{2\pi} - \frac{t}{T} \right) \right) \cos^2(y) \cos \left( \frac{\pi t}{T} \right) - \frac{2\pi y}{T},
\end{align}
where $\hat{\Psi} = 10$, and $T = 5$ is the duration of integration, after which time the analytic solution is equal to the initial condition.
\TODO{@Hilary: in \citet{chen2017} equation 41, there is a term $\left( L_x / (2\pi) \right)^2$, but this is equal to one since $L_x = 2 \pi$!  What's that about?}

The initial tracer density $\phi$ is defined as the sum of two Gaussian hills,
\begin{align}
	\phi = \phi_1(x,y) + \phi_2(x,y) \text{ ,}
\end{align}
where an individual hill $\phi_i$ is given by
\begin{align}
	\phi_i(x,y) = \phi_0 \exp \left( -b \left( \Mag{\vect{x} - \vect{x}_i} \right)^2 \right)
\end{align}
where $\phi_0 = \SI{0.95}{\kilo\gram\per\meter\cubed}$ and $b = 5$.
The initial tracer field has two hills centred at 
\begin{align}
	(x_1,y_1) &= (5 \pi /6, 0) \text{ ,} \\
	(x_2,y_2) &= (7 \pi /6, 0) \text{ .}
\end{align}
Tests were performed using the cubicFit and highOrderFit schemes using uniform meshes and meshes with distortions similar to a cubed-sphere mesh.
Uniform meshes comprise square cells so that $\Delta x \mathbin{:} \Delta y = 1\mathbin{:}1$.
Distorted meshes modify the corresponding uniform mesh using a coordinate transform,
\begin{align}
	x^\star = x, \quad
	y^\star = 
	\begin{cases}
		\pi \frac{y - f}{\pi - 2f} & \enskip \text{if $y \geq f$,} \\
		\pi \frac{y - f}{\pi + 2f} & \enskip \text{otherwise,} \\
	\end{cases}
%
\intertext{where $(x,y)$ are the physical coordinates, $(x^\star,y^\star)$ are the computational coordinates, and $f$ is given by}
%
	f = 
	\begin{cases}
		\tan(\ang{30}) \left( \frac{\pi}{4} - \Mag{x} \right) & \text{if $\Mag{x} \leq \frac{\pi}{2}$,} \\
		\tan(\ang{30}) \left( \Mag{x} - \frac{3\pi}{4} \right) & \text{otherwise.} \\
	\end{cases}
\end{align}

\begin{figure}
	\centering
	%
	\includegraphics{thesis/highOrderFit/deformationPlane/fig-meshes.pdf}
	%
	\caption{A distorted mesh on a Cartesian plane that has distortions similar to a cubed-sphere mesh.  This coarse distorted mesh has $60 \times 30$ cells such that $\Delta x = \ang{6}$.}
	\label{fig:highOrderFit:deformationPlane:mesh}
\end{figure}

Figure~\ref{fig:highOrderFit:deformationPlane:mesh} illustrates a resulting distorted mesh with $60 \times 30$ cells.
The classical fourth-order Runge–Kutta time-stepping scheme is used for both cubicFit and highOrderFit transport schemes, and tests are integrated using a time-step chosen for each mesh so that the maximum Courant number is about \num{0.4}.  
\begin{figure}
	\centering
	%
	\input{highOrderFit/deformationPlane/convergence}
	%
	\caption{Numerical convergence of the deformational flow test on a Cartesian plane.
	$\ell_2$ (equation~\ref{eqn:l2-error}) and $\ell_\infty$ errors (equation~\ref{eqn:linf-error}) are marked at mesh spacings between \ang{6} and \ang{0.375} using cubicFit and highOrderFit transport schemes on uniform and distorted meshes.}
	\label{fig:highOrderFit:deformationPlane:convergence}
\end{figure}

To measure numerical convergence, a range of mesh spacings are chosen between $\Delta x = \ang{0.375}$ and $\Delta x = \ang{6}$, and $\ell_2$ and $\ell_\infty$ errors are calculated for the cubicFit and highOrderFit schemes on each mesh (figure~\ref{fig:highOrderFit:deformationPlane:convergence}).
Similar to the results of deformational flow on a sphere in section~\ref{sec:cubicFit:deformationSphere}, both the cubicFit scheme and the highOrderFit scheme are slow to converge on coarser meshes.
At finer mesh spacings, the cubicFit scheme achieves second-order convergence and the highOrderFit scheme achieves third-order convergence.
For both schemes, errors are slightly larger switching from a uniform mesh to a distorted mesh, but the order of convergence remains unchanged.

Results presented in this chapter demonstrate that, assuming a sufficiently smooth tracer, the highOrderFit transport scheme achieves third-order convergence or higher, irrespective of mesh distortions or the choice of velocity field.
Thanks to its high-order convergence, the highOrderFit scheme is more accurate than the cubicFit scheme on all but the coarsest meshes.

