\null\vfil
\begin{abstract}
% word limit 300
Numerical weather and climate models are using increasingly fine meshes that resolve small-scale, steeply-sloping terrain.
Terrain-following meshes become highly distorted above such steep slopes, degrading the numerical balance between the pressure gradient and gravity.
Furthermore, existing models often prefer dimensionally-split transport schemes for their computational efficiency, but such schemes can suffer from splitting errors above steep slopes.
The cut cell method offers an alternative that avoids most mesh distortions, but arbitrarily small cut cells can impose severe time-step constraints on explicit transport schemes.
This thesis makes three contributions to improve atmospheric simulations, particularly in the vicinity of steeply-sloping terrain.

First, a multidimensional transport scheme is formulated to obtain accurate solutions on arbitrary, highly-distorted meshes.
Stability conditions derived from a von Neumann analysis are imposed during model initialisation to obtain stability and improve accuracy near steeply-sloping lower boundaries.
Reconstruction calculations depend upon the mesh only, needing just one vector multiply per face per time-stage irrespective of the velocity field.
The scheme achieves second-order convergence across a series of tests using highly-distorted terrain-following meshes and cut cell meshes.
The scheme is extended to achieve high-order accuracy on distorted meshes without increasing the computational cost during integration.

Second, a new type of mesh is designed to avoid severe mesh distortions associated with terrain-following meshes and avoids severe time-step constraints associated with cut cells.
Numerical experiments compare the new mesh with terrain-following and cut cell meshes, revealing that the new mesh simultaneously achieves an accurate balance between the pressure gradient and gravity, and avoids severe time-step constraints. 

Third, a new two-dimensional test case is proposed that excites the Lorenz computational mode.
The new test is used to compare results from a nonhydrostatic model with Lorenz staggering with those from a model variant with a newly-developed generalised Charney–Phillips staggering for arbitrary meshes.
\end{abstract}
\vfil

