\chapter{Discussion}
\label{ch:discussion}

Atmospheric models are using increasingly fine horizontal mesh spacings that resolve steep slopes in terrain resulting in highly-distorted meshes and increased numerical errors.
This thesis makes four contributions to reduce numerical errors for flows over steep slopes.
First, we presented a new multidimensional method-of-lines transport scheme, cubicFit, that enforces stability conditions derived from a von Neumann stability analysis to make the scheme stable over steep terrain on highly-distorted, arbitrary meshes.
The scheme has a low computational cost at runtime, requiring only $m$ multiplies per face per time-stage using a stencil with $m$ cells.
Stability condition calculations are pre-computed during model initialisation since they depend upon the mesh geometry only.
A new transport test case with a terrain-following velocity field was formulated which reveals that numerical transport errors are primarily due to misalignment of the velocity field with mesh layers and not simply mesh distortions.
In all tests, compared to the multidimensional linear upwind scheme, the cubicFit scheme is more stable and more accurate.
The cubicFit transport scheme is second-order convergent on two-dimensional meses over steeply sloping terrain, cubed sphere meshes and hexagonal icosahedral meshes, irrespective of the velocity field or mesh distortions.

\TODO{Second, highOrderFit...}

Third, we proposed a new type of mesh, the slanted cell mesh, for representing the atmosphere above steeply sloping terrain.
The slanted cell mesh is designed to avoid severe mesh distortions associated with terrain-following meshes, and to avoid severe time-step constraints associated with arbitrarily small cut cells.
In a test of a stratified atmosphere at rest, spurious circulations were reduced by switching from the highly-distorted basic terrain-following mesh to the more uniform slanted cell mesh.
A new test case was formulated to challenge transport schemes over a steeply sloping lower boundary.
Unlike the multidimensional linear upwind scheme, the cubicFit scheme is numerically stable over these very steep slopes, and the test reveals that the slanted cell mesh permits longer time-steps than cut cells, with time-steps comparable to terrain-following meshes.

\TODO{Fourth, C--P...}

\TODO{some ideas for future work...}

\textbf{cubicFit/highOrderFit}

\TODO{
\begin{itemize}
\item I'm still unaware of mathematical theory that tells use: for a given mesh geometry, what polynomial should we attempt to use as an approximation? this all seems wrapped up with linear algebra: full-rank matrices and condition numbers etc, but also have to think in terms of the stability of the discretisation?
\item put flux stabilisation from cubicFit into highOrderFit: expect that it would stabilise schaerAdvect for coarse BTF meshes
\item an accurate transport scheme walks the line between damping and instability.  \citet{devendran2017} note that the multipliers that go into the weighted least squares fit are extra degrees of freedom that could be optimised (somehow).  We do make some attempt at this by adjusting multipliers in cubicFit stabilisation procedure, but I'm not aware of any previous research into how to optimise these weights.  We want to approach $A=1$ from below to guarantee stability with minimal damping.
\item highOrderFit coefficients sometimes mildly violate the cubicFit stability constraints.  Could we do a new 1D stability analysis using the cell-average approach of highOrderFit instead of the point-wise approach of cubicFit that would provide slightly more permissive stability constraints?
\item highOrderFit for spherical geometries throws up issues of calculating high-order moments for spherical polyhedra.  \citet{sjoegreen2012} might provide us with some inspiration here?  but really we need an extension of the methods of \citet{tuzikov2003}.
\end{itemize}
}

\textbf{slanted cells}

\TODO{be interesting to try and get mesh refinement near lower boundary.  we've also talked about a TF/cut/slanted cell blend: use TF to capture slope flows, but get better pressure gradients and horizontal flows aloft.  Haven't given any of this much thought, though}
