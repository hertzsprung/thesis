\section{High-order transport over mountains}
\label{sec:highOrderFit:schaerAdvectSmooth}

Section~\ref{sec:cubicFit:schaerAdvect} presented a more challenging variant of the standard two-dimensional transport test formulated by \citet{schaer2002} to assess the numerical accuracy of the cubicFit scheme transporting a tracer horizontally above very steep slopes represented by highly-distorted terrain-following meshes.
In this section, we use the original test as specified by \citet{schaer2002} that has shallower slopes than the test presented in section~\ref{sec:cubicFit:schaerAdvect}.
We make only one modification to the original test, choosing a smoother initial tracer to allow high-order convergence to be achieved \citep{holdaway2008}.
This test is used to measure the order of convergence of the cubicFit scheme and the highOrderFit scheme on undistorted cut cell meshes and distorted terrain-following meshes.

The test follows the specification by \citet{schaer2002}, with the same domain size and boundary conditions as in section~\ref{sec:cubicFit:schaerAdvect}.
The mountain profile is given by equation~\eqref{eqn:cubicFit:schaerAdvect:mountain} and the prescribed velocity field is given by equation~\eqref{eqn:cubicFit:schaerAdvect:velocity}.
As originally specified by \citet{schaer2002}, the peak mountain height $h_0 = \SI{3}{\kilo\meter}$, and the transition from zero flow near the ground to uniform horizontal flow aloft occurs between $z_1 = \SI{4}{\kilo\meter}$ and $z_2 = \SI{5}{\kilo\meter}$.
All other parameters relating to the mountain profile and velocity field are the same as those given in section~\ref{sec:cubicFit:schaerAdvect}.

The tracer density is given by equation~\eqref{eqn:cubicFit:schaerAdvect:tracer} and is centred at $(x_0, z_0) = (\SI{-50}{\kilo\meter}, \SI{9}{\kilo\meter})$.  In order to allow high-order convergence to be achieved, the exponent $\rhoexp = 4$ such that the $\cos^\rhoexp$ hill has $\rhoexp - 1$ continuous derivatives \citep{holdaway2008}.
All other tracer parameters are the same as those given in section~\ref{sec:cubicFit:schaerAdvect}.

Tests are integrated for \SI{10000}{\second} using the classical fourth-order Runge–Kutta time-stepping scheme with both cubicFit and highOrderFit transport schemes, and a time-step chosen for each mesh so that the maximum Courant number is about \num{0.4}.  
The analytic solution at $t = \SI{10000}{\second}$ is centred at $(x_0, z_0) = (\SI{50}{\kilo\meter}, \SI{9}{\kilo\meter})$.
To measure the order of convergence of the cubicFit scheme and the highOrderFit scheme, tests are performed using mesh spacings between $\Delta x = \SI{5000}{\meter}$ and $\Delta x = \SI{250}{\meter}$.
The vertical mesh spacing $\Delta z$ is chosen so that $\Delta x \mathbin{:} \Delta z = 2\mathbin{:}1$ to match the original test specification by \citet{schaer2002}.

\begin{figure}
	\centering
	%
	\input{highOrderFit/schaerAdvectSmooth/convergence}
	%
	\caption{Numerical convergence in a test transporting a $\cos^4$ tracer horizontally over mountains.
	$\ell_2$ (equation~\ref{eqn:l2-error}) and $\ell_\infty$ errors (equation~\ref{eqn:linf-error}) are marked at mesh spacings between $\Delta x = \SI{5000}{\meter}$ and $\Delta x = \SI{250}{\meter}$ using cubicFit and highOrderFit transport schemes on basic terrain-following and cut cell meshes.}
	\label{fig:highOrderFit:schaerAdvectSmooth:convergence}
\end{figure}

The $\ell_2$ and $\ell_\infty$ errors are measured on a series of basic terrain-following meshes and cut cell meshes (figure~\ref{fig:highOrderFit:schaerAdvectSmooth:convergence}).
The cubicFit transport scheme achieves second-order convergence on cut cell meshes, and the scheme achieves third-order convergence on basic terrain-following meshes.
It is perhaps surprising that higher-order convergence is achieved when the mesh is distorted and flow is misaligned with the mesh.
Further tests using very fine meshes might confirm whether third-order convergence is maintained.
We might expect that the cubicFit scheme tends towards second-order convergence at finer mesh spacings, otherwise the cubicFit scheme would become more accurate on basic terrain-following meshes than cut cell meshes!
\TODO{perhaps discuss highOrderFit before cubicFit}

The highOrderFit transport scheme also achieves third-order convergence on basic terrain-following meshes, with error measures very similar to those obtained with the cubicFit scheme.
The highOrderFit scheme achieves fourth-order convergence on cut cell meshes.
While we have not formally analysed the order of accuracy of the highOrderFit scheme, we might expect to obtain fourth-order convergence under these ideal test conditions: first, the highOrderFit scheme uses a cubic reconstruction; second, cut cell meshes are undistorted away from the lower boundary and, third, the horizontal flow is aligned with the cut cell mesh.

At coarser mesh spacings, the cubicFit scheme is more accurate than the highOrderFit scheme.
It should be possible to increase the accuracy of the highOrderFit scheme by adjusting the multiplier values used in the least-squares fit, but this has not been explored.
At finer mesh spacings, the highOrderFit scheme becomes more accurate than the cubicFit scheme thanks to its higher order of convergence.

Further numerical experiments were performed using steeper slopes represented by basic terrain-following meshes, but the highOrderFit scheme produced instabilities above the steepest slopes.
The stabilisation procedure used in the cubicFit scheme has not been implemented in the highOrderFit scheme, but examination of the stencils reveals that, using the highOrderFit scheme, some stencil weights violate the stability conditions given in equation~\eqref{eqn:cubicFit:stability}.
It is likely that the selective removal of some high-order terms for particularly distorted stencils could make the highOrderFit scheme stable on highly-distorted meshes.

This series of horizontal transport tests demonstrates that, under favourable conditions with a sufficiently smooth tracer and uniform flow, the highOrderFit scheme is capable of fourth-order and third-order convergence on undistorted and distored meshes respectively.
The tests also demonstrated a surprising result that the cubicFit scheme is seemingly capable of third-order convergence only on distorted, terrain-following meshes.
In the next section, we evaluate the highOrderFit scheme on distorted and undistorted meshes using a more challenging, deformational flow.
