\section{Deformational flow on a plane}
\label{sec:highOrderFit:deformationPlane}

\TODO{motivate and introduce test case}

Following \citet{chen2017}, the domain is defined on a rectangular $x$--$y$ plane that is $2\pi$ wide and $\pi$ tall.  The domain is periodic in the $x$ direction with no normal flow imposed at the upper and lower boundaries.
The discrete velocity field is defined using the streamfunction,
\begin{align}
	\Psi = \frac{\hat{\Psi}}{T} \sin^2 \left( 2 \pi \left( \frac{x}{2\pi} - \frac{t}{T} \right) \right) \cos^2(y) \cos \left( \frac{\pi t}{T} \right) - \frac{2\pi y}{T},
\end{align}
where $\hat{\Psi} = 10$, and $T = 5$ is the duration of integration, after which time the analytic solution is equal to the initial condition.
\TODO{in \citet{chen2017} equation 41, there is a term $\left( L_x / (2\pi) \right)^2$, but this is equal to one since $L_x = 2 \pi$!  What's that about?}

The initial tracer density $\phi$ is defined as the sum of two Gaussian hills,
\begin{align}
	\phi = \phi_1(x,y) + \phi_2(x,y) \text{ ,}
\end{align}
where an individual hill $\phi_i$ is given by
\begin{align}
	\phi_i(x,y) = \phi_0 \exp \left( -b \left( \Mag{\vect{x} - \vect{x}_i} \right)^2 \right)
\end{align}
where $\phi_0 = \SI{0.95}{\kilo\gram\per\meter\cubed}$ and $b = 5$.
The initial tracer field has two hills centred at 
\begin{align}
	(x_1,y_1) &= (5 \pi /6, 0) \text{ ,} \\
	(x_2,y_2) &= (7 \pi /6, 0) \text{ .}
\end{align}
Tests were performed using the cubicFit and highOrderFit schemes using uniform meshes and meshes with distortions similar to a cubed-sphere mesh.
Uniform meshes comprise square cells so that $\Delta x \mathbin{:} \Delta y = 1\mathbin{:}1$.
Distorted meshes modify the corresponding uniform mesh using a coordinate transform,
\begin{align}
	X = x, \quad
	Y = 
	\begin{cases}
		\pi \frac{y - f}{\pi - 2f} & \enskip \text{if $y \geq f$,} \\
		\pi \frac{y - f}{\pi + 2f} & \enskip \text{otherwise,} \\
	\end{cases}
%
\intertext{where \TODO{$(x,y)$ are the ??? coordinates, $(X,Y)$ are the ??? coordinates, and $f$ is given by}}
%
	f = 
	\begin{cases}
		\tan(\ang{30}) \left( \frac{\pi}{4} - \Mag{x} \right) & \text{if $\Mag{x} \leq \frac{\pi}{2}$,} \\
		\tan(\ang{30}) \left( \Mag{x} - \frac{3\pi}{4} \right) & \text{otherwise.} \\
	\end{cases}
\end{align}
Figure~\ref{fig:highOrderFit:deformationPlane:mesh} illustrates a resulting distorted mesh with $60 \times 30$ cells.
The classical fourth-order Runge–Kutta time-stepping scheme is used for both cubicFit and highOrderFit transport schemes, and tests are integrated using a time-step chosen for each mesh so that the maximum Courant number is about \num{0.4}.  

\begin{figure}
	\centering
	%
	\includegraphics{thesis/highOrderFit/deformationPlane/fig-meshes.pdf}
	%
	\caption{A distorted mesh on a Cartesian plane that has distortions similar to a cubed-sphere mesh.  This coarse distorted mesh has $60 \times 30$ cells such that $\Delta x = \ang{6}$.}
	\label{fig:highOrderFit:deformationPlane:mesh}
\end{figure}

\begin{figure}
	\centering
	%
	\input{highOrderFit/deformationPlane/convergence}
	%
	\caption{Numerical convergence of the deformational flow test on a Cartesian plane.
	$\ell_2$ (equation~\ref{eqn:l2-error}) and $\ell_\infty$ errors (equation~\ref{eqn:linf-error}) are marked at mesh spacings between \ang{6} and \ang{0.375} using cubicFit and highOrderFit transport schemes on uniform and distorted meshes.}
	\label{fig:highOrderFit:deformationPlane:convergence}
\end{figure}

To measure numerical convergence, a range of mesh spacings are chosen between $\Delta x = \ang{0.375}$ and $\Delta x = \ang{6}$, and $\ell_2$ and $\ell_\infty$ errors are calculated for the cubicFit and highOrderFit schemes on each mesh (figure~\ref{fig:highOrderFit:deformationPlane:convergence}).
Similar to the results of deformational flow on a sphere in section~\ref{sec:cubicFit:deformationSphere}, both the cubicFit scheme and the highOrderFit scheme are slow to converge on coarser meshes.
At finer mesh spacings, the cubicFit scheme achieves second-order convergence and the highOrderFit scheme achieves third-order convergence.
\TODO{we don't see fourth-order convergence in this test, why not?}
For both schemes, errors are slightly larger switching from a uniform mesh to a distorted mesh, but the order of convergence remains unchanged.
For a given mesh, the highOrderFit scheme is more accurate than the cubicFit scheme except at the coarsest mesh spacings.

