\section{High-order finite volume formulation}
\label{sec:highOrderFit:scheme}

Averaging the flux-form transport equation \eqref{eqn:advection} over a volume $\vol$ and using Gauss's divergence theorem,
\begin{align}
	\frac{1}{\vol} \int_\vol \frac{\partial \phi}{\partial t} d\vol = - \frac{1}{\vol} \int_{\partial \vol} \phi \vect{u} \cdot \unitn dA \label{eqn:highOrder:integrated-advection}
\end{align}
where $\unitn$ is the outward unit normal vector.
Using the method of lines, the time derivative is discretised using the classical fourth-order Runge–Kutta time-stepping scheme \citep[p. 53]{durran2013}, and the spatial discretisation is described next.
For a polygonal cell with faces $f$ equation~\eqref{eqn:highOrder:integrated-advection} becomes
\begin{align}
	\frac{1}{\vol} \int_\vol \frac{\partial \phi}{\partial t} d\vol = - \frac{1}{\vol} \sum_f \int_{\area_f} \phi \vect{u} \cdot \unitn d\area_f \label{eqn:highOrder:advection}
\end{align}
where $\area_f$ is the area of face $f$.
If $\phi$ is a sufficiently smooth field then it can be approximated to $P$-order accuracy by replacing $\phi$ with a polynomial interpolant $\psi$,
\begin{align}
	\psi = \sum_{\Mag{\vect{p}} \leq P} a_{\vect{p}} \left(\vect{x} - \vect{x}_0 \right)^\vect{p} \label{eqn:highOrder:interpolant}
\end{align}
where $a_\vect{p}$ are unknown polynomial coefficients, $\vect{x}_0$ is a fixed position, and $P$ is the total polynomial order.
Note that we use multi-index notation such that $\Mag{\vect{p}} = p_1 + \ldots + p_n$ and
\begin{align}
	a_{\vect{p}} \left( \vect{x} - \vect{x}_0 \right)^\vect{p} = a_\vect{p} \prod_{d=1}^D \left( x_d - x_{0_d} \right)^{p_d}
\end{align}
where $D$ is the number of physical dimensions.
As an example, the exponents $\vect{p}$ in two dimensions $(x, y)$ with $\Mag{\vect{p}} \leq 1$ are $(0, 0)$, $(1, 0)$ and $(0, 1)$, hence the two-dimensional polynomial interpolant for a total polynomial order $P = 1$ is
\begin{align}
	\psi = a_{0,0} + a_{1,0} \left( x - x_0 \right) + a_{0,1} \left( y - y_0 \right) \text{ .}
\end{align}
Replacing $\phi$ in \eqref{eqn:highOrder:advection} with $\psi$ in \eqref{eqn:highOrder:interpolant} we obtain an expression for the face flux\footnote{Equation~\eqref{eqn:highOrder:face-flux} assumes that there is little variation in the velocity field $\vect{u}$ at the grid scale, such that $\vect{u}$ can be approximated by a velocity $\uf{}$ that is constant over the face $f$.
Such an assumption is justifiable for atmospheric flows that are predominantly large-scale \citep{methven-hoskins1999}.},
\begin{align}
	\int_\area \phi \vect{u} \cdot \unitn d\area = \uf \cdot \unitn \sum_{\Mag{\vect{p}} \leq P} a_{\vect{p}} \moment_\area^\vect{p} \label{eqn:highOrder:face-flux}
\end{align}
where $\moment_\area^\vect{p} = \int_\area \left( \vect{x} - \vect{x}_0 \right)^\vect{p} d \area$ is the $\vect{p}$-th moment of area $\area$.
Therefore, the face flux can be calculated by finding the the polynomial coefficients $a_\vect{p}$.

Following the same approach as the cubicFit transport scheme, taking a total polynomial order $P = 3$ gives 9 polynomial terms with polynomial coefficients calculated using the same upwind-biased stencil.
For every cell in the stencil we require that the average of the polynomial integrated over a cell volume equals the cell average value,
\begin{align}
	\volave{\psi} = \volave{\phi} \label{eqn:highOrder:equal-volumes}
%
\intertext{where the average over volume $\vol$ is}
%
	\volave{\psi} = \frac{1}{\vol} \int_\vol \psi d\vol \text{ .} \label{eqn:highOrder:volume-average}
\end{align}
Using equations~\eqref{eqn:highOrder:interpolant} and \eqref{eqn:highOrder:volume-average} we can rewrite equation~\eqref{eqn:highOrder:equal-volumes} as
\begin{align}
	\frac{1}{\moment_\vol^\vect{0}} \sum_{\Mag{\vect{p}} \leq P} a_\vect{p} \moment_\vol^\vect{p} = \volave{\phi}
\end{align}
where $\moment_\vol^\vect{p} = \int_\vol \left( \vect{x} - \vect{x}_0 \right)^\vect{p} d\vol$ is the $\vect{p}$-th moment of volume $\vol$, and the zeroth moment $\moment_\vol^\vect{0}$ is the volume.
For $m$ polynomial terms and a stencil with $n$ cells, we calculate a face flux by choosing $\vect{x}_0$ to be the position of the face centre, then we write the linear system
\begin{align}
	\begin{bmatrix}
		\moment_{\vol_1}^{\vect{p}_1}/\moment_{\vol_1}^\vect{0} & \cdots & \moment_{\vol_1}^{\vect{p}_m}/\moment_{\vol_1}^\vect{0} \\
		\vdots & & \vdots \\
		\moment_{\vol_n}^{\vect{p}_1}/\moment_{\vol_n}^\vect{0} & \cdots & \moment_{\vol_n}^{\vect{p}_m}/\moment_{\vol_n}^\vect{0}
	\end{bmatrix}
	\begin{bmatrix}
		a_{\vect{p}_1} \\
		\vdots \\
		a_{\vect{p}_m}
	\end{bmatrix}
	=
	\begin{bmatrix}
		\langle \phi \rangle_{\vol_1} \\
		\vdots \\
		\langle \phi \rangle_{\vol_n}
	\end{bmatrix} \label{eqn:highOrder:moment-matrix}
\end{align}
which can be written as
\begin{align}
	\vect{B} \vect{a} = \bm{\phi} \text{ .}
\end{align}
where $\vect{B}$ is the stencil matrix, which is constructed using only the mesh geometry.
The highOrderFit scheme generates stencils using the same procedure as the cubicFit scheme.
Assuming a stencil comprises at least as many cells as there are polynomial coefficients then $n \geq m$ and the matrix equation can be solved using a least-squares approach to find the unknown coefficients $\vect{a}$.

To obtain a stable transport scheme, we follow the approach of the cubicFit scheme by introducing multipliers $\vect{m}$ to obtain
\begin{align}
	\vect{\tilde{B}} \vect{a} = \vect{m} \cdot \bm{\phi}
\end{align}
where $\vect{\tilde{B}} = \vect{M} \vect{B}$ and $\vect{M} = \mathrm{diag}(\vect{m})$.
The upwind cell and downwind cell have multipliers $m_u = 2^{10}$ and $m_d = 2^{10}$ respectively, and all peripheral points have multipliers $m_p = 1$.

The calculation of high-order cell volume moments and surface moments are required by equations~\eqref{eqn:highOrder:moment-matrix} and~\eqref{eqn:highOrder:face-flux} respectively.  These volume and surface moments can be calculated exactly using the method of \citet{tuzikov2003}.
We follow their method but, in order to avoid any degenerate triangles, we introduce a centre point shared by all triangles instead of triangulating polygons with only existing vertices.

While the highOrderFit transport scheme uses a total polynomial order $P = 3$ for stencils in the domain interior, a total polynomial order $P = 1$ is used for stencils near the boundary having fewer than 12 cells.
This reduction in total polynomial order ensures that matrix equations are not underconstrained.
This thesis does not assess the accuracy of the highOrderFit scheme near boundaries, and so the more sophisticated boundary treatment implemented in the cubicFit scheme has not been implemented in the highOrderFit scheme.

