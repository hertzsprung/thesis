\section{A two-dimensional standing waves test case}
\label{sec:cp:arakawaKonor}

\TODO{
\begin{itemize}
	\item Plot: horizontal edgeGrading, vertical edgeGrading
	\item Plot: conservation of internal energy and total energy time series, ExnerFoamH and ExnerFoamCP, uniform mesh, horizontal edgeGrading, vertical edgeGrading
	\item Conclusion: the test excites the Lorenz computational mode with ExnerFoamH
	\item Conclusion: ExnerFoamCP eliminates computational mode
	\item Conclusion: edgeGrading reveals lack of conservation, due to advective form transport scheme?
\end{itemize}
}

\TODO{a little intro.  say again that all this is based on the standing waves test case by \citet{arakawa-konor1996}.}

The domain is \SI{30}{\kilo\meter} high and \SI{600}{\kilo\meter} wide between the outermost faces, and the mesh spacing is $\Delta x = \SI{10}{\kilo\meter}$ and $\Delta z = \SI{1}{\kilo\meter}$.  The lower boundary is flat with no mountain profile.
The upper and lower boundaries are no normal flow, and the domain is horizontally periodic.

The initial potential temperature profile is the sum of a stably-stratified profile and a grid-scale perturbation near the ground.
The stably-stratified profile has $\theta(z = 0) = \SI{250}{\kelvin}$ and a constant static stability with Brunt-V\"ais\"al\"a frequency $N = \SI{0.02}{\per\second}$.
The potential temperature perturbation $\theta'$ is defined as
\begin{subequations}
\label{eqn:cp:arakawaKonor:theta-perturbation}
\begin{align}
	\theta' &= \begin{cases} S \theta'_0 \sin(\frac{2 \pi x}{\lambda}) & \text{if } \Mag{x} \leq \frac{\lambda}{2} \\
		0 & \text{otherwise} \\
	\end{cases} \\
\shortintertext{where}
	S &= \begin{cases}
		-1 & \text{if } \SI{1}{\kilo\meter} <= z < \SI{2}{\kilo\meter} \\
		1 & \text{if } \SI{2}{\kilo\meter} <= z < \SI{3}{\kilo\meter} \\
		0 & \text{otherwise} \\
	\end{cases}
\end{align}
\end{subequations}
and the maximum amplitude $\theta'_0 = \SI{0.5}{\kelvin}$ and the wavelength $\lambda = \SI{100}{\kilo\meter}$.
Using a Lorenz staggering, this arrangement produces grid-scale waves in the central region of the domain in two layers near the ground (figure~\ref{fig:cp:arakawaKonor:theta_diff:initial}).
Using a generalised Charney--Phillips staggering, the perturbation is non-zero on the lowest two interior mesh layers above the lower boundary (not shown).
The definition given by equation~\eqref{eqn:cp:arakawaKonor:theta-perturbation} ensures that the potential temperature perturbation integrated over the domain is zero.

At the upper and lower boundaries, zero gradients are imposed on the potential temperature field for the Lorenz model variant; for the Charney--Phillips model variant, fixed potential temperature values are prescribed using equation~\ref{eqn:cp:schaerWaves:thermal-profile}.
The Exner function of pressure is calculated so that it is in discrete hydrostatic balance.

A sponge layer is added to the upper \SI{10}{\kilo\meter}.  The damping function is given by
\begin{align}
	\mu(z) &= \begin{cases}
		\overline{\mu} \sin^2 \left( \frac{\pi}{2} \frac{z - z_B}{H - z_B} \right) & \text{if } z \geq z_B \\
		0 & \text{otherwise} \\
	\end{cases}
\end{align}
where $\overline{\mu} = \SI{1.2}{\per\second}$ is the damping coefficient, $z_B = \SI{20}{\kilo\meter}$ is the bottom of the sponge layer and $H = \SI{30}{\kilo\meter}$ is the top of the domain.
The sponge layer is only active on faces whose normal is vertical so that it damps vertical momentum only.

The test is integrated forward by 48 hours using a time-step of $\Delta t = \SI{25}{\second}$.

\begin{figure}
	\centering
	\begin{subfigure}{0.38\textwidth}
		\vspace*{1.1em}
		\includegraphics{arakawaKonor-uniform-lorenz/0/theta_diff.pdf}
		\caption{Initial perturbation with a Lorenz staggering}
		\label{fig:cp:arakawaKonor:theta_diff:initial}
	\end{subfigure}
	\begin{subfigure}{0.3\textwidth}
		\includegraphics{arakawaKonor-uniform-lorenz/172800/theta_diffS.pdf}
		\caption{Lorenz solution}
		\label{fig:cp:arakawaKonor:theta_diff:lorenz}
	\end{subfigure}
	\begin{subfigure}{0.3\textwidth}
		\vspace*{1.1em}
		\includegraphics{arakawaKonor-uniform-cp/172800/theta_diffS.pdf}
		\caption{Generalised Charney--Phillips solution}
		\label{fig:cp:arakawaKonor:theta_diff:cp}
	\end{subfigure}
%
\vspace{0.5em}
\includegraphics[height=5in,angle=270]{arakawaKonor-uniform-lorenz/arakawaKonor-initial-theta_diff-colorBar.eps}
%
	\caption{Differences in potential temperature for the standing waves test case.
	(\subcaptionref{fig:cp:arakawaKonor:theta_diff:initial}) a grid-scale potential temperature perturbation near the surface is added to an initial, stably-stratified profile.
The difference between the initial, unperturbed, stably-stratified potential temperature profile and the final solution are shown using
	(\subcaptionref{fig:cp:arakawaKonor:theta_diff:lorenz}) the Lorenz model variant, and 
	(\subcaptionref{fig:cp:arakawaKonor:theta_diff:cp}) the generalised Charney--Phillips model variant.
Only the lowest \SI{20}{\kilo\meter} in the central region of the domain is shown.  The entire domain is \SI{600}{\kilo\meter} wide and \SI{30}{\kilo\meter} high.
	}
	\label{fig:cp:arakawaKonor:theta_diff}
\end{figure}

\TODO{discuss uniform results}

\TODO{introduce edgeGrading meshes: should we specify these with a Jacobian rather than OpenFOAM's edgeGrading notation?}
