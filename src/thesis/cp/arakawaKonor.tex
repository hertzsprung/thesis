\section{A two-dimensional standing waves test case}
\label{sec:cp:arakawaKonor}

Having verified that the generalised Charney--Phillips formulation produces a reasonable mountain waves solution, we must also verify that the formulation is free from the Lorenz computational mode.
Since existing tests are not well-suited for nonhydrostatic model evaluation, we design a new, two-dimensional standing waves test case, based on the original specification by \citet{arakawa-konor1996}.
Results are compared between Lorenz and generalised Charney--Phillips model variants.
To explore the applicability of the generalised Charney--Phillips formulation to arbitrary vertical meshes, we also compare results between distorted and undistorted meshes.

The domain is \SI{30}{\kilo\meter} high and \SI{600}{\kilo\meter} wide between the outermost faces, and the mesh spacing is $\Delta x = \SI{10}{\kilo\meter}$ and $\Delta z = \SI{1}{\kilo\meter}$.  The lower boundary is flat with no mountain profile.
The upper and lower boundaries are no normal flow, and the domain is horizontally periodic.

The initial potential temperature profile is the sum of a stably-stratified profile and a grid-scale perturbation near the ground.
The stably-stratified profile has $\theta(z = 0) = \SI{250}{\kelvin}$ and a constant static stability with Brunt-V\"ais\"al\"a frequency $N = \SI{0.02}{\per\second}$.
The potential temperature perturbation $\theta'$ is defined as
\begin{subequations}
\label{eqn:cp:arakawaKonor:theta-perturbation}
\begin{align}
	\theta' &= \begin{cases}
		S \theta'_0 \sin(\frac{2 \pi x}{\lambda}) & \enskip \text{if $\Mag{x} \leq \frac{\lambda}{2}$,} \\
		0 & \enskip \text{otherwise,} \\
	\end{cases} \\
\shortintertext{where $S$ is given by}
	S &= \begin{cases}
		-1 & \enskip \text{if $\SI{1}{\kilo\meter} \leq z < \SI{2}{\kilo\meter}$,} \\
		1 & \enskip \text{if $\SI{2}{\kilo\meter} \leq z < \SI{3}{\kilo\meter}$,} \\
		0 & \enskip \text{otherwise,} \\
	\end{cases}
\end{align}%
\end{subequations}
with the maximum amplitude $\theta'_0 = \SI{0.5}{\kelvin}$ and the wavelength $\lambda = \SI{100}{\kilo\meter}$.
Using a Lorenz staggering, this arrangement produces grid-scale waves in the central region of the domain in two adjacent layers near the ground (figure~\ref{fig:cp:arakawaKonor:theta_diff:uniform:initial}).
Using a generalised Charney--Phillips staggering, the perturbation is non-zero at the lowest two interior mesh layers above the lower boundary (not shown).
The definition given by equation~\eqref{eqn:cp:arakawaKonor:theta-perturbation} ensures that the potential temperature perturbation integrated over the domain is zero.
Using the Lorenz model variant, potential temperature is transported using the linearUpwind scheme.
Using the Charney--Phillips model variant, potential temperature is transported in advective form (equation~\ref{eqn:cp:advection}).

At the upper and lower boundaries zero gradients are imposed on the potential temperature field for the Lorenz model variant; for the Charney--Phillips model variant, fixed potential temperature values are prescribed using equation~\ref{eqn:cp:schaerWaves:thermal-profile}.
The Exner function of pressure is calculated so that it is in discrete hydrostatic balance \rev{with the perturbed potential temperature field}.

A sponge layer is added to the upper \SI{10}{\kilo\meter}.  The damping function is given by
\begin{align}
	\mu(z) &= \begin{cases}
		\overline{\mu} \sin^2 \left( \frac{\pi}{2} \frac{z - z_B}{H - z_B} \right) & \text{if $z \geq z_B$,} \\
		0 & \text{otherwise,} \\
	\end{cases}
\end{align}
where $\overline{\mu} = \SI{1.2}{\per\second}$ is the damping coefficient, $z_B = \SI{20}{\kilo\meter}$ is the bottom of the sponge layer and $H = \SI{30}{\kilo\meter}$ is the top of the domain.
The sponge layer is active \rev{only on entirely horizontal faces so that only vertical momentum is damped}.


\begin{figure}
	\centering
	\begin{subfigure}{\textwidth}
		\phantomsubcaption\label{fig:cp:arakawaKonor:theta_diff:uniform:initial}
		\phantomsubcaption\label{fig:cp:arakawaKonor:theta_diff:uniform:lorenz}
		\phantomsubcaption\label{fig:cp:arakawaKonor:theta_diff:uniform:cp}
		\phantomsubcaption\label{fig:cp:arakawaKonor:theta_diff:hEdgeGrading:initial}
		\phantomsubcaption\label{fig:cp:arakawaKonor:theta_diff:hEdgeGrading:lorenz}
		\phantomsubcaption\label{fig:cp:arakawaKonor:theta_diff:hEdgeGrading:cp}
		\includegraphics{thesis/cp/arakawaKonor/fig-theta_diff.pdf}
	\end{subfigure}
%
	\caption{
	Differences in potential temperature for the standing waves test case.
	On the uniform mesh and the horizontally tilted mesh,
	(\subcaptionref{fig:cp:arakawaKonor:theta_diff:uniform:initial},
	\subcaptionref{fig:cp:arakawaKonor:theta_diff:hEdgeGrading:initial})
	a grid-scale potential temperature perturbation near the surface is added to an initial, stably-stratified profile;
the difference between the initial, unperturbed, stably-stratified potential temperature profile and the final solution are shown using
	(\subcaptionref{fig:cp:arakawaKonor:theta_diff:uniform:lorenz},
	\subcaptionref{fig:cp:arakawaKonor:theta_diff:hEdgeGrading:lorenz})
	the Lorenz model variant, and 
	(\subcaptionref{fig:cp:arakawaKonor:theta_diff:uniform:cp},
	\subcaptionref{fig:cp:arakawaKonor:theta_diff:hEdgeGrading:cp})
	the generalised Charney--Phillips model variant.
	The largest differences in potential temperature on the horizontally tilted mesh using the generalised Charney--Phillips model variant is about \SI{1.5}{\kelvin}, which is not representable by the colour scale used here.
Only the lowest \SI{22}{\kilo\meter} in the central region of the domain is shown.  The entire domain is \SI{600}{\kilo\meter} wide and \SI{30}{\kilo\meter} high.
	}
	\label{fig:cp:arakawaKonor:theta_diff}
\end{figure}

The test is integrated forward by 48 hours using a time-step of $\Delta t = \SI{25}{\second}$.
The initial potential temperature perturbation generates gravity waves that spread rapidly through the domain.
In addition to these gravity waves, using the Lorenz model variant, a grid-scale standing wave slowly spreads vertically, occupying the entire depth of the atmosphere (figure~\ref{fig:cp:arakawaKonor:theta_diff:uniform:lorenz}), and the initial perturbation, though weakened in amplitude, persists throughout the duration of the simulation.
No standing waves are produced by the generalised Charney--Phillips model variant (figure~\ref{fig:cp:arakawaKonor:theta_diff:uniform:cp}).
When the mesh is refined vertically such that $\Delta z = \SI{500}{\meter}$ then the initial potential temperature perturbation is no longer at the grid scale, occupying four mesh layers rather than two.
On this refined mesh, neither the Lorenz model variant nor the generalised Charney--Phillips model variant produce any standing waves (not shown).
Since they are produced only by an initial grid-scale potential temperature perturbation using the Lorenz model variant, we conclude that the grid-scale standing waves are a spurious feature excited by the Lorenz computational mode.

\begin{figure}
	\centering
	\begin{subfigure}{\textwidth}
		\phantomsubcaption\label{fig:cp:arakawaKonor:edgeGradedMesh:h}
		\phantomsubcaption\label{fig:cp:arakawaKonor:edgeGradedMesh:v}
		\centering
		\includegraphics{thesis/cp/arakawaKonor/fig-meshes.pdf}
	\end{subfigure}
%
	\caption{Distorted meshes used for the standing waves test with
	(\subcaptionref{fig:cp:arakawaKonor:edgeGradedMesh:h}) horizontally tilted and
	(\subcaptionref{fig:cp:arakawaKonor:edgeGradedMesh:v}) vertically tilted surfaces.}
	\label{fig:cp:arakawaKonor:edgeGradedMesh}
\end{figure}

To assess the suitability of the generalised Charney--Phillips formulation for arbitrary vertical meshes, we perform the same standing waves test case on rectangular domains with tilted horizontal surfaces (figure~\ref{fig:cp:arakawaKonor:edgeGradedMesh:h}) and tilted vertical surfaces (figure~\ref{fig:cp:arakawaKonor:edgeGradedMesh:v}).
To allow for periodic lateral boundaries, each mesh is split into left and right blocks of equal size, with the right block mirroring the left.
For the left-hand block of the horizontally tilted mesh, horizontal surfaces are distorted such that the ratio of \rev{minimum} and maximum vertical edge lengths at $x = \SI{-300}{\kilo\meter}$ is $1 \mathbin{:} 16$, and the ratio at $x = \SI{0}{\kilo\meter}$ is $16 \mathbin{:} 1$.
The vertically tilted mesh is \rev{constructed in a} similar manner, with a ratio of minimum and maximum horizontal edge lengths at $z = \SI{0}{\kilo\meter}$ of $81 \mathbin{:} 100$, and a ratio at $z = \SI{30}{\kilo\meter}$ of $100 \mathbin{:} 81$.

Using the horizontally tilted mesh, large-scale responses are produced by both the Lorenz model variant (figure~\ref{fig:cp:arakawaKonor:theta_diff:hEdgeGrading:lorenz}) and generalised Charney--Phillips model variant (figure~\ref{fig:cp:arakawaKonor:theta_diff:hEdgeGrading:cp}), and these large-scale responses are very different from the small-scale gravity waves produced by the Lorenz and generalised Charney--Phillips models using the uniform mesh (figures~\ref{fig:cp:arakawaKonor:theta_diff:uniform:lorenz} and \ref{fig:cp:arakawaKonor:theta_diff:uniform:cp} respectively).
Using the Lorenz model variant and the horizontally tilted mesh, a vertical tripole structure is seen in the final potential temperature field, with some grid-scale features visible near the centre of the domain throughout the depth of the atmosphere (figure~\ref{fig:cp:arakawaKonor:theta_diff:hEdgeGrading:lorenz}).
A similar solution is produced by the Lorenz model variant using the vertically tilted mesh (not shown).
In contrast to the solutions produced by the Lorenz model variant, using the generalised Charney--Phillips model variant and the horizontally tilted mesh produces a solution that is everywhere too warm (figure~\ref{fig:cp:arakawaKonor:theta_diff:hEdgeGrading:cp}).
Using the generalised Charney--Phillips model variant and the vertically tilted mesh, the atmosphere warms rapidly before the model becomes unstable after about \num{2.5} hours.
Note that, since the initial potential temperature perturbation is defined in Cartesian coorinates and not relative to model layers, the discrete initial potential temperature perturbation differs slightly between the uniform mesh (figure~\ref{fig:cp:arakawaKonor:theta_diff:uniform:initial}), horizontally tilted mesh (figure~\ref{fig:cp:arakawaKonor:theta_diff:hEdgeGrading:initial}) and vertically tilted mesh (not shown).
However, we do not expect this slight initial difference to result in such dramatically different solutions.

\begin{figure}
	\centering
	\input{cp/arakawaKonor/conservation}
	%
	\caption{Total normalised energy changes for the standing waves test case using Lorenz and generalised Charney--Phillips model variants on a uniform mesh, a horizontally tilted mesh and a vertical tilted mesh.
	Energy changes are negligible using the Lorenz model variant on all meshes, and using the generalised Charney--Phillips model variant on the uniform mesh.}
	\label{fig:cp:arakawaKonor:conservation}
\end{figure}

To better examine the thermal errors produced by the generalised Charney--Phillips model variant, we calculate total energy change over time, normalised by the initial total energy, where the total energy is the sum of the kinetic energy, potential energy and internal energy (figure~\ref{fig:cp:arakawaKonor:conservation}).
Energy changes are negligible using the Lorenz model variant on all meshes, and using the generalised Charney--Phillips model variant on the uniform mesh.
Using the generalised Charney--Phillips model and the horizontally tilted mesh, total energy increases linearly with time, which corresponds with the spurious warm atmosphere seen in figure~\ref{fig:cp:arakawaKonor:theta_diff:hEdgeGrading:cp}.
Using the vertically tilted mesh, a rapid increase in total energy is observed at about $t=2.5$ hours, just before the model becomes unstable.
Since the generalised Charney--Phillips model variant transports potential temperature using an advective-form scheme and not a flux-form scheme, it is likely that the observed energy changes and associated potential temperature errors are due to a lack of conservation on distorted meshes.

Here we have presented a new standing waves test case that has been used to clearly excite the Lorenz computational mode, and we have demonstrated that the generalised Charney--Phillips formulation is free from the Lorenz computational mode on a uniform mesh.
The generalised Charney--Phillips model variant suffers from inaccurate or unstable solutions on distorted meshes, but we expect that a more accurate transport scheme could avoid such a severe lack of conservation and improve solutions on distorted meshes.

