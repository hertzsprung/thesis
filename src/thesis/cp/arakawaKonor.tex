\section{A new test case to excite the Lorenz computational mode}
\label{sec:cp:arakawaKonor}

\TODO{
\begin{itemize}
	\item Test: 2D vertical slice with initial $\theta$ perturbation, ExnerFoamH and ExnerFoamCP
	\item Meshes: uniform, edgeGrading (sloping horizontal surfaces, sloping vertical faces)
	\item Plot: horizontal edgeGrading, vertical edgeGrading
	\item Plot: initial theta\_diff, final theta\_diff for ExnerFoamH and ExnerFoamCP
	\item Plot: conservation of internal energy and total energy time series, ExnerFoamH and ExnerFoamCP, uniform mesh, horizontal edgeGrading, vertical edgeGrading
	\item Conclusion: the test excites the Lorenz computational mode with ExnerFoamH
	\item Conclusion: ExnerFoamCP eliminates computational mode
	\item Conclusion: edgeGrading reveals lack of conservation... is it due to the advective form transport of $\theta$?  next test will try to provide an answer
\end{itemize}
}

% domain: 600km (between outermost faces), 30km high
% 60x30 cells, dx=10km, dz=1km
% periodic lateral boundaries
% background theta
% theta perturbation (so that the integral of theta over the domain is unaltered)
% sponge layer

\begin{figure}
	\centering
	\begin{subfigure}{0.38\textwidth}
		\includegraphics{arakawaKonor-uniform-lorenz/0/theta_diff.pdf}
		\caption{Initial perturbation}
		\label{fig:cp:arakawaKonor:theta_diff:initial}
	\end{subfigure}
	\begin{subfigure}{0.3\textwidth}
		\includegraphics{arakawaKonor-uniform-lorenz/172800/theta_diff.pdf}
		\caption{Lorenz solution}
		\label{fig:cp:arakawaKonor:theta_diff:lorenz}
	\end{subfigure}
	\begin{subfigure}{0.3\textwidth}
		\includegraphics{arakawaKonor-uniform-cp/172800/theta_diff.pdf}
		\caption{Generalised Charney--Phillips solution}
		\label{fig:cp:arakawaKonor:theta_diff:cp}
	\end{subfigure}
	\caption{Differences in potential temperature for the standing waves test case.
	\TODO{}
Only the lowest \SI{20}{\kilo\meter} in the central region of the domain is shown.  The entire domain is \SI{600}{\kilo\meter} wide and \SI{30}{\kilo\meter} high.
	}
	\label{fig:cp:arakawaKonor:theta_diff}
\end{figure}
