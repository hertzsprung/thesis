\section{Transport over a mountainous lower boundary}
\label{sec:slanted:mountainAdvect}

The two-dimensional tests performed in chapter~\ref{ch:cubicFit} transported tracers positioned well above the terrain surface.  Here we formulate a new test, positioning the tracer at the ground in order to assess the accuracy of transport schemes immediately above a mountainous lower boundary.  Results using the cubicFit scheme are compared with the linearUpwind scheme on basic terrain-following, cut cell and slanted cell meshes.
The test presents a particular challenge to transport schemes as they must transport the tracer through arbitrarily small cut cells and distorted slanted cells.

The domain size and mountain profile is the same as those in the horizontal tracer transport test in section~\ref{sec:cubicFit:schaerAdvect}, with a mesh spacing of $\Delta x = \SI{1000}{\meter}$ and $\Delta z = \SI{500}{\meter}$.
In order to present the most challenging test on slanted cell meshes, and so thin cells remain near mountain peaks.
Cell edges in the central region of the domain are shown in figure~\ref{fig:slanted:mountainAdvect:meshes} for each of the three mesh types.
Cells in the BTF mesh are highly distorted over steep slopes (figure~\ref{fig:slanted:mountainAdvect:meshes:btf}) while the cut cell mesh (figure~\ref{fig:slanted:mountainAdvect:meshes:cutCell}) and slanted cell mesh (figure~\ref{fig:slanted:mountainAdvect:meshes:slantedCell}) are orthogonal everywhere except for cells nearest the ground.

\begin{figure}
	\centering
	\begin{subfigure}{\textwidth}
		\phantomsubcaption\label{fig:slanted:mountainAdvect:meshes:btf}
		\phantomsubcaption\label{fig:slanted:mountainAdvect:meshes:cutCell}
		\phantomsubcaption\label{fig:slanted:mountainAdvect:meshes:slantedCell}
		\includegraphics{thesis/slanted/mountainAdvect/fig-meshes.pdf}
	\end{subfigure}
%
	\caption{Two dimensional $x$-$z$ meshes created with the (\subcaptionref{fig:slanted:mountainAdvect:meshes:btf}) basic terrain-following,
	(\subcaptionref{fig:slanted:mountainAdvect:meshes:cutCell}) cut cell, and
	(\subcaptionref{fig:slanted:mountainAdvect:meshes:slantedCell}) slanted cell methods, used for the tracer transport tests in section~\ref{sec:slanted:mountainAdvect}.  Cell edges are marked by thin black lines.  The peak mountain height $h_0 = \SI{5}{\kilo\meter}$.
The velocity field is the same for all mesh types with streamlines marked on each panel by thick red lines.
The velocity field (equation~\ref{eqn:streamfunc-btf}) follows the lower boundary and becomes entirely horizontal above $H_1 = \SI{10}{\kilo\meter}$.
Only the lowest \SI{10}{\kilo\meter} for the central region of the domain is shown.  The entire domain is \SI{300}{\kilo\meter} wide and \SI{25}{\kilo\meter} high.}
	\label{fig:slanted:mountainAdvect:meshes}
\end{figure}

A velocity field is prescribed using equation~\eqref{eqn:streamfunc-btf} so that the flow follows the terrain at the surface and becomes entirely horizontal above $H_1 = \SI{10}{\kilo\meter}$.
The value of $H_1$ is chosen to be much smaller than the domain height $H$ in equation~\eqref{eqn:btf} so that flow crosses the surfaces of the BTF mesh.
This is evident in figure~\ref{fig:slanted:mountainAdvect:meshes:btf} where the the velocity streamlines are tangential to the mesh only at the ground.
The flow is deliberately misaligned with the BTF, cut cell and slanted cell meshes away from the ground (figure~\ref{fig:slanted:mountainAdvect:meshes}) to ensure that flow always crosses mesh surfaces in order to challenge the transport schemes.

\begin{table}
	\centering
	\input{slanted/mountainAdvect/timesteps}
%
	\caption{Time-steps (\si{\second}) for the two-dimensional transport test over a mountainous lower boundary.  The time-steps were chosen so that the maximum Courant number was between \num{0.36} and \num{0.46}.}
	\label{tab:slanted:mountainAdvect:timesteps}
\end{table}

The tracer is defined again by equation~\eqref{eqn:cubicFit:schaerAdvect:tracer} but is now positioned at the ground with $(x_0, z_0) = (\SI{-50}{\kilo\meter}, \SI{0}{\kilo\meter})$ with half-widths $A_x = \SI{25}{\kilo\meter}$ and $A_z = \SI{10}{\kilo\meter}$.
Tests are integrated forward for \SI{10000}{\second}.  The time-step was chosen for each mesh so that the maximum Courant number was about \num{0.4} (table~\ref{tab:slanted:mountainAdvect:timesteps}).
An analytic solution at \SI{10000}{\second} is obtained by calculating the new horizontal position of the tracer using equation~\eqref{eqn:cubicFit:tfAdvect:trajectory}.
By solving this equation we find that \(x(t=\SI{10000}{\second}) = \SI{6244.087}{\meter}\) when $h_0 = \SI{5}{\kilo\meter}$.

The tracer density boundary conditions are the same as those in section~\ref{sec:cubicFit:schaerAdvect}.
Since the cubicFit transport scheme uses values at boundaries with Dirichlet boundary conditions, the cubicFit scheme uses only inlet boundary values in this test case.

\begin{figure}
	\centering
	\includegraphics{thesis/slanted/mountainAdvect/fig-tracer.pdf}
	%
	\caption{Evolution of the tracer in the two-dimensional transport test over a mountainous lower boundary.  The tracer is transported to the right over the wave-shaped terrain.  Tracer contours are every \SI{0.1}{\kilo\gram\per\meter\cubed}.  The result obtained using the cubicFit scheme on the basic terrain-following mesh is shown at $t=\SI{0}{\second}$, $t=\SI{5000}{\second}$ and $t=\SI{10000}{\second}$ with solid black contours. The analytic solution at $t=\SI{10000}{\second}$ is shown with dotted contours.
	The shaded box indicates the region that is plotted in figure~\ref{fig:slanted:mountainAdvect:errors}.}
	\label{fig:slanted:mountainAdvect:tracer}
\end{figure}

Three series of tests were performed using similar configurations.  The first series uses a peak mountain height of $h_0 = \SI{5}{\kilo\meter}$ to examine errors on different mesh types using the two transport schemes.
The second series varies the peak mountain height to examine the sensitivity of the transport schemes to mesh distortions.
The third series verifies accuracy at Courant numbers close to the limit of stability, and examines the longest stable time-step for different mesh types.

\subsection{A comparison of numerical accuracy between mesh types and transport schemes}
For the first series of tests with $h_0 = \SI{5}{\kilo\meter}$, tracer contours at the initial time $t=\SI{0}{\second}$, half-way time $t=\SI{5000}{\second}$, and end time $t=\SI{10000}{\second}$ are shown in figure~\ref{fig:slanted:mountainAdvect:tracer} using the cubicFit scheme on the BTF mesh.  As apparent at $t=\SI{5000}{\second}$, the tracer is distorted by the terrain-following velocity field as it passes over the mountain as expected, and its original shape is restored once it has cleared the mountain by $t=\SI{10000}{\second}$.
Slight errors are apparent at $t = \SI{10000}{\second}$ when the numerical solution marked with solid contour lines is compared with the analytic solution marked with dotted contour lines.

\begin{figure}
	\begin{subfigure}{\textwidth}
		\centering
		\phantomsubcaption\label{fig:slanted:mountainAdvect:errors:linearUpwind-btf}
		\phantomsubcaption\label{fig:slanted:mountainAdvect:errors:linearUpwind-cutCell}
		\phantomsubcaption\label{fig:slanted:mountainAdvect:errors:linearUpwind-slantedCell}
		\phantomsubcaption\label{fig:slanted:mountainAdvect:errors:cubicFit-btf}
		\phantomsubcaption\label{fig:slanted:mountainAdvect:errors:cubicFit-cutCell}
		\phantomsubcaption\label{fig:slanted:mountainAdvect:errors:cubicFit-slantedCell}
		%
		\includegraphics{thesis/slanted/mountainAdvect/fig-error.pdf}
	\end{subfigure}

	\caption{Tracer contours at $t=\SI{10000}{\second}$ for the two-dimensional tracer transport tests over a mountainous lower boundary.  A region in the lee of the mountain is plotted corresponding to the shaded area in figure~\ref{fig:slanted:mountainAdvect:tracer}.
	Results are presented on BTF, cut cell and slanted cell meshes (shown in figure~\ref{fig:slanted:mountainAdvect:meshes}) using the linearUpwind and cubicFit transport schemes.  The numerical solutions are marked by solid black lines.  The analytic solution is marked by dotted lines.  Contours are every \SI{0.1}{\kilo\gram\per\meter\cubed}.}
	\label{fig:slanted:mountainAdvect:errors}
\end{figure}

Numerical errors are more clearly revealed by subtracting the analytic solution from the numerical solution.
Error fields are compared between BTF, cut cell and slanted cell meshes using the linearUpwind scheme (figures~\ref{fig:slanted:mountainAdvect:errors:linearUpwind-btf},
\ref{fig:slanted:mountainAdvect:errors:linearUpwind-cutCell} and
\ref{fig:slanted:mountainAdvect:errors:linearUpwind-slantedCell} respectively) and the cubicFit scheme (figures~\ref{fig:slanted:mountainAdvect:errors:cubicFit-btf},
\ref{fig:slanted:mountainAdvect:errors:cubicFit-cutCell} and
\ref{fig:slanted:mountainAdvect:errors:cubicFit-slantedCell} respectively).
Results are least accurate using the linearUpwind scheme on the slanted cell mesh (figure~\ref{fig:slanted:mountainAdvect:errors:linearUpwind-slantedCell}) with the final tracer being slightly distorted.
The $\ell_\infty$ error magnitude is reduced by using the linearUpwind scheme on the cut cell mesh (figure~\ref{fig:slanted:mountainAdvect:errors:linearUpwind-cutCell}), but the shape of the error remains the same.
On the BTF mesh (figure~\ref{fig:slanted:mountainAdvect:errors:cubicFit-btf}), cut cell mesh (figure~\ref{fig:slanted:mountainAdvect:errors:cubicFit-cutCell}) and slanted cell mesh (figure~\ref{fig:slanted:mountainAdvect:errors:cubicFit-slantedCell}), the cubicFit scheme is more accurate than the linearUpwind scheme.

\subsection{Numerical stability and numerical accuracy with increasingly steep slopes}

\begin{figure}
	\centering
	\input{slanted/mountainAdvect/l2ByMountainHeight}
%
	\caption{Error measures for the two-dimensional tracer transport tests over a mountainous lower boundary.  Peak mountain heights $h_0$ are from \SIrange{0}{6}{\kilo\meter}.  Results are compared on BTF, cut cell and slanted cell meshes using the linearUpwind and the cubicFit schemes.  At $h_0 = \SI{0}{\kilo\meter}$ the terrain is entirely flat and the BTF, cut cell and slanted cell meshes are identical.  At $h_0 = \SI{6}{\kilo\meter}$ the linearUpwind scheme is unstable on the slanted cell mesh.}
	\label{fig:slanted:mountainAdvect:l2ByMountainHeight}
\end{figure}

To further examine the performance of the cubicFit scheme in the presence of steep terrain, a second series of tests were performed in which the peak mountain height was varied from \SIrange{0}{6}{\kilo\meter} keeping all other parameters constant.
Results were obtained on BTF, cut cell and slanted cell meshes using the linearUpwind scheme and cubicFit scheme.  Again, the time-step was chosen for each test so that the maximum Courant number was about \num{0.4} (table~\ref{tab:slanted:mountainAdvect:timesteps}).  The $\ell_2$ error was calculated by subtracting the analytic solution from the numerical solution (figure~\ref{fig:slanted:mountainAdvect:l2ByMountainHeight}).
Note that the analytic solution is a function of mountain height, with the tracer travelling farther over higher mountains due to non-divergent flow through a narrower channel.
In all cases, error increases with increasing mountain height because steeper slopes lead to greater mesh distortions.
Errors are identical for a given transport scheme when $h_0 = \SI{0}{\kilo\meter}$ and the ground is entirely flat because the BTF, cut cell and slanted cell meshes are identical.
The linearUpwind scheme is unstable on the slanted cell mesh with a peak mountain height $h_0 = \SI{6}{\kilo\meter}$ despite using a Courant number of \inputval{mountainAdvect-h0-slantedCell-1000-6000m-linearUpwind/co}\unskip.
The cubicFit scheme yields stable results in all tests, and cubicFit is more accurate than linearUpwind in all tests.

\subsection{Numerical stability limits of the cubicFit transport scheme}

\begin{figure}
	\begin{subfigure}{\textwidth}
		\centering
		\input{slanted/mountainAdvect/maxdt}
		\phantomsubcaption\label{fig:slanted:mountainAdvect:maxdt:dt}
		\phantomsubcaption\label{fig:slanted:mountainAdvect:maxdt:co}
	\end{subfigure}
	%
	\caption{(\subcaptionref{fig:slanted:mountainAdvect:maxdt:dt}) Longest stable time-steps, $\Delta t_\mathrm{max}$, and 
	(\subcaptionref{fig:slanted:mountainAdvect:maxdt:co}) largest stable maximum Courant numbers, $\max(\mathrm{Co})$, for the two-dimensional tracer transport test over a mountainous lower boundary.  Results were obtained on basic terrain-following, cut cell and slanted cell meshes at mesh spacings between $\Delta x = \SI{5000}{\meter}$ and $\Delta x = \SI{250}{\meter}$.  The largest stable maximum Courant numbers were calculated from the corresponding longest stable time-steps using equation~\eqref{eqn:co}.}
	\label{fig:slanted:mountainAdvect:maxdt}
\end{figure}

A final series of tests were performed to determine the stability limit of the cubicFit scheme with the two-stage Heun time-stepping scheme (equation~\ref{eqn:heun}).
The tracer was transported on BTF, slanted cell and cut cell meshes with a variety of mesh spacings between $\Delta x = \SI{5000}{\meter}$ and $\Delta x = \SI{125}{\meter}$.  $\Delta z$ was chosen so that a constant aspect ratio is preserved such that $\Delta x \mathbin{:} \Delta z = 2 \mathbin{:} 1$.
For each test, the time-step was adjusted in order to find the largest stable time-step, $\Delta t_\mathrm{max}$ (figure~\ref{fig:slanted:mountainAdvect:maxdt:dt}).
BTF meshes permit the longest time-steps of all three mesh types since cells are almost uniform in volume.  As expected, the longest stable time-step scales linearly with BTF mesh spacing.
There is no such linear scaling on cut cell meshes because these meshes can have arbitrarily small cells.  The time-step constraints on cut cell meshes are the most severe of the three mesh types.  Slanted cell meshes have a slightly more stringent time-step constraint than BTF meshes but still exhibit similar linear scaling with mesh spacing.

Figure~\ref{fig:slanted:mountainAdvect:maxdt:co} presents the largest stable maximum Courant numbers, $\max(\mathrm{Co})$, which were calculated by substituting $\Delta t = \Delta t_\mathrm{max}$ into equation~\eqref{eqn:co}.
On basic terrain-following meshes, the maximum Courant number tends towards about \num{1.3} with finer mesh spacings.
No such trend is found on cut cell or slanted cell meshes.
Cut cell meshes permit the largest maximum Courant numbers of around \num{3}, but the largest stable time-steps on cut cell meshes are still smaller than corresponding time-steps on basic terrain-following and slanted cell meshes.

This thesis focuses on the spatial discretisation of the cubicFit scheme, but the stability limit depends also upon the choice of time-stepping.  We have not calculated a theoretical Courant number limit, although such an analysis should be possible using the techniques of \citet{baldauf2008}.

This new test case demonstrates that the cubicFit transport scheme is more accurate than the linearUpwind scheme on all meshes, and only the cubicFit scheme can achieve stable results on slanted cell meshes with very steep slopes.
The slanted cell method exhibits a time-step constraint that scales linearly with mesh spacing, and slanted cells avoid severe time-step constraints associated with arbitrarily small cut cells.
Next, we incorporate the cubicFit transport scheme into a model of the fully compressible Euler equations.
