\section{Discretisation of the fully compressible Euler equations}
\label{sec:slanted:exnerFoamH}

The finite volume model of the fully compressible Euler equations is taken from \citet{weller-shahrokhi2014}, given by

\begin{subequations}
\begin{align}
	\text{Momentum} &\ &\  	\frac{\partial \rho \vect{u}}{\partial t} + \nabla \cdot \rho \vect{u}\otimes\vect{u} &= \rho \vect{g} - c_p \rho \theta \nabla \Pi - \mu \rho \vect{u} \label{eqn:exnerFoam:momentum} \\
	\text{Continuity} &\ &\	\frac{\partial \rho}{\partial t} + \nabla \cdot \rho \vect{u} &= 0 \label{eqn:exnerFoam:cont} \\
	\text{Thermodynamic equation} &\ &\ \frac{\partial \rho \theta}{\partial t} + \nabla \cdot \rho \vect{u} \theta &= 0 \label{eqn:exnerFoam:theta} \\
	\text{Ideal gas law} &\ &\ \Pi^{(1 - \kappa)/\kappa} &= \frac{R \rho \theta}{p_0} \label{eqn:exnerFoam:state}
\end{align}
\end{subequations}
where \(\rho\) is the density, \(\vect{u}\) is the velocity field, \(\vect{g}\) is the gravitational acceleration, \(c_p\) is the heat capacity at constant pressure, \(\theta = T \left(p_0/p\right)^\kappa\) is the potential temperature, \(T\) is the temperature, \(p\) is the pressure, \(p_0 = \SI{1000}{\hecto\pascal}\) is a reference pressure, \(\Pi = \left(p / p_0 \right)^\kappa\) is the Exner function of pressure, and \(\kappa = R/c_p\) is the gas constant to heat capacity ratio.  \(\mu\) is a damping function that can be used to absorb momentum near the upper boundary.

The model uses the C-grid staggering in the horizontal and the Lorenz staggering in the vertical such that $\theta$, $\rho$ and $\Pi$ are stored at cell centroids and the covariant component of velocity at cell faces.  The model is configured without Coriolis forces.

Acoustic and gravity waves are treated implicitly and transport terms are treated explicitly.
The trapezoidal implicit treatment of fast waves and the Hodge operator suitable for non-orthogonal meshes are described in the appendix to \citet{shaw-weller2016}.
To avoid time-splitting errors between transport and fast waves, transport is time-stepped using a three-stage, second-order Runge-Kutta scheme.
The transport terms of the momentum equation \eqref{eqn:exnerFoam:momentum} and thermodynamic equation \eqref{eqn:exnerFoam:theta} are discretised in flux form using the cubicFit transport scheme.

This model is suitable for arbitrary meshes and includes a curl-free pressure gradient formulation.
In the next section, we use this model to compare the accuracy of pressure gradient calculations using terrain-following, cut cell and slanted cell meshes.
