\chapter{Introduction}

\TODO{it would be nice to have a citation for each desirable properties demonstrating what can happen if that property is absent
\begin{description}
	\item[Conservation]{Conserve mass for climate runs, conserve moisture to avoid Met Office's `eternal fountain'}
	\item[Numerical stability]{grid-point storms when CFL is violated -- not really the fault of the advection scheme, though!}
	\item[Numerical diffusion]{\citet{schaer2002, kent2014} show it happens with horizontal flow on TF meshes.  \citet{hoinka-zaengl2004} show it happens for moisture and momentum advection.  How about spurious smoothing of fronts? stable isotopes of water, important for paleoclimate simulations \citep{cauquoin-risi2017}}
	\item[Numerical dispersion/spurious oscillations]{Spurious oscillations found in many variables: in OLAM \citep{walko-avissar2008b}, MC2 \citep{schaer2002} and MM5 \citep{hoinka-zaengl2004}.  Also problematic in ocean/sea-ice modelling \citep{naughten2017}.}
	\item[Better than first-order convergence]{\citep{staniforth-thuburn2012}}
	\item[Grid imprinting]{\citep{staniforth-thuburn2012} Anyone talk about it in the context of advection? \citet{thuburn2014} mentions it but his advection scheme shows little sign of grid imprinting}
\end{description}
Can use OLAM as an example for motivating numerical methods for arbitrary meshes: think about choice of horizontal and vertical meshes and their combination}

