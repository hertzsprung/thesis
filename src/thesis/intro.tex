\chapter{Introduction}

Atmospheric models are using increasingly fine mesh spacing to resolve small-scale processes and improve weather and climate forecasts \citep{wedi2014}.
These finer meshes resolve small-scale, steeply-sloping terrain that is poorly represented by traditional terrain-following meshes \citep{schaer2002}, motivating research into alterative vertical meshes including improved terrain-following meshes \citep{schaer2002,klemp2011} and cut cell meshes \citep{jaehn2015,yamazaki2016}, and improved numerical methods \citep{zaengl2012,steppeler-klemp2017}.

Terrain-following meshes have been in widespread operational use since atmospheric models first included a numerical representation of terrain, with the basic terrain following mesh having been formulated by \citet{galchen-somerville1975a}.
Basic terrain following meshes distort every model layer above sloping terrain, with only the upper boundary being entirely horizontal.
These mesh distortions become more severe with increasingly steep slopes, reducing the numerical accuracy of transport schemes and pressure gradient calculations in particular.

\citet{schaer2002} found that lower-order transport schemes are inaccurate in the presence of mesh distortions, with the transported tracer exhibiting spurious numerical diffusion and grid-scale oscillations.
In a mesoscale model forecast over the Alps, numerical diffusion spoiled the solution of water vapour and relative vorticity fields near the tropopause where vertical gradients are strong \citep{hoinka-zaengl2004}.
A model intercomparison by \citet{kent2014} highlighted the detrimental effects of horizontal-vertical splitting in idealised three-dimensional transport tests over terrain.
\TODO{numerical diffusion errors:
Spurious oscillations found in many variables: in OLAM \citep{walko-avissar2008b}, MC2 \citep{schaer2002} and MM5 \citep{hoinka-zaengl2004}.  Also problematic in ocean/sea-ice modelling \citep{naughten2017}}

\TODO{pressure gradient errors}

\TODO{BTF alternatives: SLEVE etc, cut cells (small cells and alleviations)}

%Furthermore, with increasingly fine mesh spacing in global atmospheric models, quasi-uniform spherical meshes become more computationally efficient than traditional latitude-longitude meshes by avoiding the pole singularity \citep{staniforth-thuburn2012}, and mesh refinement and adaptive mesh generation become increasingly necessary to limit the growing number of degrees of freedom \TODO{good citations}.
% some nice text in MC#5

\TODO{our solutions: cubicFit+highOrderFit, slanted cells, C--P}

% more mesh flexibility is desirable (see refs in MC#3)

% cubicFit
% - should I mention horizontal meshes, too?  relevant only for cubicFit
% slanted cells
% highOrderFit
% C--P

